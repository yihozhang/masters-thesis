\chapter{Optimizing Non-relational E-matching}\label{chapter/nonrel-em}

E-matching is an important procedure for many \egraph based applications,
 yet it is slow.
In a typical application of equality saturation,
 \ematching can take 60-90\% of the overall run time \citep{egg}.
In the work presented in my bachelor's thesis, 
 my collaborators and I proposed a relational approach to \ematching, 
 dubbed relational \ematching \citep{relational-ematching, relational-ematching-thesis}.
In particular,
 we made e-matching orders of magnitude faster, 
 proved theoretical bounds of e-matching, 
 and opened the door for all kinds of
 wild optimizations that can be done with databases and e-graphs.

% I'm very proud of
% this work, not only because it is elegant, fast, and theoretically
% (worst-case) optimal, but also because it is the kind of work I'd like
% to work on: building connections across areas.

However, 
 the relational e-matching approach also has some secret pitfalls.
In particular, 
 to have the best of both efficient e-graph maintenance and efficient e-matching, 
 one has to switch back and forth 
 between the e-graph to its relational representation.
Our prototype\footnote{\url{https://github.com/egraphs-good/egg/tree/relational}}
 builds a relational database and associated indices from
 scratch for each match-apply iteration.
This is acceptable in the equality saturation setting.
E-matching and updates always alternate in batches, 
 so the cost of building the database is amortized.
Plus, 
 since building databases and indices are both linear time costs, 
 they are often subsumed by the time spent on e-matching.

However, 
 what if e-matching is not run in batches? 
Or what if all the e-matching patterns are quite simple 
 and the constant overhead is now a bottleneck? 
An \egraph framework can implement some fast paths for that, 
 but then there are more design questions to consider: 
Are we going to keep two implementations of e-matching? 
What kind of queries should be computed by relational
 e-matching and what by traditional e-matching? \ldots{} 
We can continue down this path and put a lot of engineering effort into
 building a practically efficient e-graph engine, or we can:

\begin{enumerate}
\item
  Start a clean-slate relational e-graph framework that handles all
  e-graph operations efficiently and forget about the graph part of an
  e-graph;
\item
  Keep the current \egraph data structure, and port some optimizations of
  relational e-matching back to \egraphs.
\end{enumerate}

In this chapter,
 I will focus on the second approach.
The rest of the thesis will focus on the first approach.
When working on relational e-matching, 
 we found an optimization to the backtracking-style classical e-matching.
Like relational e-matching, 
 it is able to improve e-matching asymptotically in some cases, 
 but it does not require transforming the input \egraph to a relational database.
And it is very simple.
For what it's worth, 
 this optimization (instead of relational e-matching) is what is currently
 implemented in egg.

In this chapter, I will describe this optimization.
But before that,
 we will go over some brief background on e-matching.
Readers are welcome to skip it if they already know what e-matching is.

\section{E-Matching}\label{e-matching}

There have been many great introductions to e-graphs and e-matching.
For example, 
 Philip Zucker gives
 a gentle introduction to 
 \egraphs \citep{zucker-egraph-1} and
 \ematching \citep{zucker-egraph-2} in Julia.
Max Willsey also wrote a very nice tutorial \citep{egg-tutorial}
 on e-graphs and egg \citep{egg}.
Basically, 
 an e-graph is a data structure that compactly represents 
 an equivalence relation and 
 e-matching is pattern matching
 on such e-graphs modulo equivalence.
Both e-matching and e-graphs are widely used in
 SMT solvers \citep{efficient-ematching}
 and equality saturation-based
 program optimizers \citep{tensat}.
A typical equality saturation-based program optimizer may
 take the majority of its time doing e-matching.

There are several algorithms proposed for e-matching.
For example,
 the one currently used in egg is based
 on the virtual machine proposed by \citet{efficient-ematching}.
The traditional backtracking-based e-matching algorithm
 does not exploit equality constraints during pattern compilation.
Equality constraints are the term we used in the relational e-matching paper
 to describe the kind of constraints that all occurrences of the same variables should be mapped
 to equivalent terms.
Those that violate the equality constraints will
 not be pruned away immediately.
For example, \(f(\alpha, g(\alpha))\)
 does not match \(f(1,g(2))\), 
 because the first \(\alpha\) is mapped to
 $1$ but the second is mapped to $2$.
The classical backtracking-based
 e-matching will still consider it though.

The relational e-matching approach instead treats an e-matching pattern
 as a kind of relational query.
From a relational query, 
 the query optimizer can easily identify all kinds of constraints, 
 including equality constraints,
 and find an efficient query plan.
As an example,
 the above pattern can be compiled to query
 \(Q(r, \alpha)\gets R_f(r, \alpha,x),R_g(x,\alpha)\), 
 and a hash join could answer this query in linear time.

\section{The optimization}\label{the-optimization}

The issue with the traditional backtracking-style e-matching 
 is that it does not take advantage of the equality constraints, 
 so it enumerates obviously unsatisfying terms.
The optimization is therefore straightforward: 
 do not enumerate terms that are obviously unsatisfying.
And this is easy, because we already know what the (only) satisfying
 term should look like!

Let's first look at the classical e-matching algorithm 
 (reproduced from \citet[Figure~3]{relational-ematching}, with a typo fix).

\begin{align*}
    \textit{match}(x,c,S) = 
                   & \{ \sigma \cup \{ x \mapsto c\} \mid \sigma \in S, x \not \in \text{dom}(\sigma)\}\ \cup\\
                   & \{ \sigma \mid \sigma \in S, \sigma(x) = c \}\\
    \textit{match}(f(p_{1}, \dots, p_{k}), c, S) = 
                   & \bigcup_{f(c_{1},\dots,c_{k})\in c}
                     \textit{match}(p_{k}, c_{k}, \dots, \textit{match}(p_{1}, c_{1}, S))
\end{align*}

It takes a pattern \(p\), an e-class \(c\), 
 and current substitutions \(S\),
 and returns the set of substitutions produced by e-matching \(p\)
 over e-class \(c\),
 such that all produced substitutions are extensions
 of some substitutions in \(S\).
The result of e-matching a pattern \(p\)
 over an e-graph is
 \(\bigcup_{c\in C} \textit{match}(p, c, \{\emptyset\})\)
 (both \citet{relational-ematching} and \citet{efficient-ematching}
 have another typo here), where \(C\) is the
 set of e-classes in the e-graph.

The algorithm is straightforward:
\begin{enumerate}
\tightlist
\item
  If the pattern is a variable, and
  \begin{enumerate}
  \tightlist
  \item
    if this variable is fresh in the domain of the substitution, then
    it's safe to extend the substitutions with \(\{x\mapsto c\}\), or
  \item
    if this variable is not fresh, we keep only these substitutions that
    are consistent with the mapping \(\{x\mapsto c\}\).
  \end{enumerate}
\item
  If the pattern is a function symbol of the form \(f(p_1,\ldots,p_k)\),
  the algorithm iterates over \(f\)-nodes \(f(c_1,\ldots, c_k)\) in the
  e-class, and fold over the sub patterns and sub e-classes with
  \(\textit{match}\), to accumulate set of valid substitutions.
\end{enumerate}

The trick is to generalize case 1.b.
In case 1.b, we know the
 substitution for a pattern is unique when the pattern is a non-fresh
 variable, but we \emph{also} know this when the variables of the pattern
 are in the domain of the substitution (i.e.,
 \(\text{fv}(p)\subseteq\text{dom}(S)\)), thanks to canonicalization.
In that case, 
 the pattern after substitution is a ground term, 
 which can be efficiently looked up in a bottom-up fashion.

To implement this idea, we lift case 1.b to the top-level of the
 algorithm.
During e-matching, 
 the algorithm will first check whether the
 free vars of the input pattern is contained in the domain of the
 substitution.
If yes, 
 then instead of looking further into the pattern,
 the algorithm will lookup the substituted term for comparison.
The following definition shows this:
\begin{align*}
    \textit{match}(p, c, S) = & \begin{cases}
        \{\sigma \mid \sigma\in S, 
                      \textit{lookup}([\sigma]e)=c\} 
        &\text{ if $\text{fv}(p)\subseteq \text{dom}(S)$ }\\
        match'(p, c, S)&\text{ o.w.}
    \end{cases}\\
    \textit{match'}(x,c,S) = 
                   & \{ \sigma \cup \{ x \mapsto c\} \mid \sigma \in S\}\\
    \textit{match'}(f(p_{1}, \dots, p_{k}), c, S) = 
                   & \bigcup_{f(c_{1},\dots,c_{k})\in c}
                     \textit{match}(p_{k}, c_{k}, \dots, \textit{match}(p_{1}, c_{1}, S))
\end{align*}
In the above definition, we also drop the check of
\(x\not\in\text{dom}(\sigma)\) for the variable case, which is
guaranteed not to happen.

As an example, consider \(f(\alpha, g(\alpha))\) again.
E-matching will
 enumerate through each \(f\)-node and bind \(\alpha\) to the first child
 of the \(f\)-node.
Here, the classical e-matching algorithm will then
 enumerate though the second child e-class of the \(f\)-node for possible
 \(g\)-nodes.
However, because \(g(\alpha)\) is a ground term after
 substituting \(\alpha\) with \(\sigma(\alpha)\), we can effectively
 lookup \(g(\alpha)\) and compare it with the e-class id of the second
 child.
The pseudocode:
\begin{Shaded}
\begin{Highlighting}[]
\CommentTok{\# classical e{-}matching}
\ControlFlowTok{for}\NormalTok{ f }\KeywordTok{in}\NormalTok{ c: }\CommentTok{\# f(a, g(a))}
  \ControlFlowTok{for}\NormalTok{ g }\KeywordTok{in}\NormalTok{ f.child2: }\CommentTok{\# g(a)}
    \ControlFlowTok{if}\NormalTok{ f.child1 }\OperatorTok{!=}\NormalTok{ g.child1:}
      \ControlFlowTok{continue}
    \ControlFlowTok{yield}\NormalTok{ \{a: f.child1\}}

\CommentTok{\# with the trick}
\ControlFlowTok{for}\NormalTok{ f }\KeywordTok{in}\NormalTok{ c: }\CommentTok{\# f(a, g(a))}
\NormalTok{  g }\OperatorTok{=}\NormalTok{ lookup(mk\_node(g, f.child1))}
  \ControlFlowTok{if}\NormalTok{ g }\KeywordTok{is} \VariableTok{None} \KeywordTok{or}\NormalTok{ g }\OperatorTok{!=}\NormalTok{ f.child2:}
    \ControlFlowTok{continue}
  \ControlFlowTok{yield}\NormalTok{ \{a: f.child1\}}
\end{Highlighting}
\end{Shaded}

Implementation-wise, egg adds a new operator to the e-matching virtual
 machine called \texttt{Lookup}.
\texttt{Lookup} (1) substitutes the
 pattern with values in the VM register to produce a ground term and (2)
 lookup the ground term in the e-graph.

\section{A Relational View of the Trick}\label{a-relational-view-of-the-trick}

How effective is this trick? To have a better understanding of this
 trick, we need to take a relational lens.
The classical e-matching can be
 viewed as a relational query plan where hash joins only index one column
 (the link between parent and child) and potentially prune using the rest
 of equality columns (the equality constraint).
At first I thought this
 optimization will make classical e-matching equivalent to some efficient
 hash join-based query plans, 
 and a efficient plan here specifically means a plan
 where the hash joins will index all the columns known to be equivalent.
But this is false.
Consider the pattern \(f(\alpha, g(\alpha,\beta))\).
The relational version of it is
 \(Q(r, \alpha,\beta)\gets R_f(r, \alpha, x),R_g(x,\alpha,\beta)\).
An efficient plan with hash joins will index both \(\alpha\) and \(x\).
However,
 our trick can't use the \(\alpha\) in \(f\) to prune the
 \(\alpha\) in \(g\), because there could be multiple satisfying
 \(g\)-nodes (due to the unbound variable \(\beta\)).
In this case, our
 optimization does not offer any speedup.

In fact, this trick can be relationally thought of as the kind of query
 optimizations that leverage functional dependencies.
In the relational representation of e-graphs,
 there is a functional dependency from the
 children columns to the id column.
For example, in relation
 \(R_f(x, c_1, c_2)\), the relational representation of binary function
 symbol \(f\), every combination of \(c_1\) and \(c_2\) uniquely
 determines \(x\) thanks to e-graph canonicalization.
Our trick uses this
 information to immediately determine the value of \(x\) once \(c_1\) and
 \(c_2\) are known, without looking at obviously unsatisfying candidates.

In the relational e-matching paper,
 we also described how we use
 functional dependency to speed up queries.
In fact, if the variable
 ordering of generic joins follows the topological order of the (acyclic)
 functional dependency,
 the run-time complexity will be worst-case
 optimal \textit{under the presence of FDs} \citep{wcoj-survey}, 
 a stronger guarantee than the original
 AGM bound \citep{agm}.
Functional dependencies are also exploited
 for query optimization in databases \citep{data-dep-for-opt-survey}.

How does this compare to relational e-matching? 
First, as we saw above,
 it is not as powerful as relational e-matching.
Moreover, 
 the graph representation has the fundamental restriction 
 that makes it very hard
 to do advanced optimizations, 
 e.g., one that uses cardinality information.
It's also limited in the kind of join it is able to
 (conceptually) perform (only hash joins).
However, 
 it integrates well
 with an existing non-relational e-graph framework,
 which relational \ematching fails to achieve.

\section{Query planning}\label{query-planning}

This trick also poses a new question for classical e-matching planning:
what visit order should one use? In the above definition of our algorithm,
 we assumed a depth-first style order of processing, but this is not
 necessary.
For example,
 after enumerating the top-level \(f\)-node in
 pattern \(f(g(\alpha), h(\alpha, \beta))\), it will be most efficient to
 enumerate the \(h\)-node and lookup \([\sigma]g(\alpha)\) later.
If however we first enumerate \(g(\alpha)\),
we still can't avoid enumerating \(h(\alpha,\beta)\) later on.

If we assume the cost of enumerating each node is the same,
 this problem can be viewed as finding the smallest connected component (CC) in the
 pattern tree that contains the root, such that the CC covers all
 distinct variables.
This is not an easy problem, and similar problems are NP-hard.
This problem can be solved 
 using dynamic programming on trees with exponential states, or can be reduced
 to an ILP problem.
However, both seem to be an overkill for realistic queries.

It is also unclear what is a practically good planning heuristic.
The one used in egg prioritizes sub-patterns with more free vars,
 but this may not be good enough: 
 consider pattern
 \(f(f(g(\alpha),\beta)),g(h(\alpha), h(\beta)))\).
 This heuristics yield the following plan for this pattern:

\begin{Shaded}
\begin{Highlighting}[]
\ControlFlowTok{for}\NormalTok{ f1 }\KeywordTok{in}\NormalTok{ c: }\CommentTok{\# f(f(g(a), g(b))), g(h(a), h(b)))}
  \ControlFlowTok{for}\NormalTok{ f2 }\KeywordTok{in}\NormalTok{ c.child1: }\CommentTok{\# f(g(a), g(b))) (2 free vars)}
    \ControlFlowTok{for}\NormalTok{ g1 }\KeywordTok{in}\NormalTok{ c.child2: }\CommentTok{\# g(h(a), h(b)) (2 free vars)}
      \ControlFlowTok{for}\NormalTok{ g2 }\KeywordTok{in}\NormalTok{ f2.child1: }\CommentTok{\# g(a) (1 free var)}
\NormalTok{        h1 }\OperatorTok{=}\NormalTok{ lookup(mk\_node(h, g2.child1) }\CommentTok{\# lookup h(a)}
        \ControlFlowTok{if}\NormalTok{ h1 }\KeywordTok{is} \VariableTok{None} \KeywordTok{or}\NormalTok{ h1 }\OperatorTok{!=}\NormalTok{ g1.child1:}
          \ControlFlowTok{continue}
        \ControlFlowTok{for}\NormalTok{ g3 }\KeywordTok{in}\NormalTok{ f2.child2: }\CommentTok{\# g(b) (1 free var)}
\NormalTok{          h2 }\OperatorTok{=}\NormalTok{ lookup(mk\_node(h, g3.child1) }\CommentTok{\# lookup h(b)}
          \ControlFlowTok{if}\NormalTok{ h2 }\KeywordTok{is} \VariableTok{None} \KeywordTok{or}\NormalTok{ h2 }\OperatorTok{!=}\NormalTok{ g1.child2:}
            \ControlFlowTok{continue}
          \ControlFlowTok{yield}\NormalTok{ \{...\}}
\end{Highlighting}
\end{Shaded}

This is complicated, but it suffices to only look at the first three loops: 
 It does a cross product over the first and the second child of
 the top-level \(f\)-node.
A good strategy here is instead to prefer
 fewer free vars, and performs the search in a depth-first search, 
 so that \(g(h(a), h(b))\) can be looked up all at once after
 \(f(g(a), g(b))\) is enumerated.
But it is not yet known if preferring fewer free vars is the
 right strategy.
Moreover, realistic patterns
 tend to be small and simple, 
 so cases like the above may be rare.

\section{Miscellaneous}\label{miscellaneous}

This chapter is adapted 
 from my blog post 
 \textit{A Trick that Makes Classical E-Matching Faster} \citep{nonrelational-ematching-post}.
I thank Max and Philip for their valuable discussions and comments.
The presented trick stems from a Pull Request\footnote{\url{https://github.com/egraphs-good/egg/pull/74}}
 that tries to improve e-matching for ground terms.
In hindsight, 
 a variant of the proposed improvement targeting multi-patterns
 had been discussed in \citet{efficient-ematching}
 but was lost in egg's original e-matching implementation.
Compared to that Pull Request, which
 only looks up ground terms, this optimization generalizes it by also
 looking up terms that are grounded after substitution.
Philip came up with this idea independently 
 as well\footnote{\url{https://github.com/egraphs-good/egg/pull/74\#issuecomment-818833367}}.
