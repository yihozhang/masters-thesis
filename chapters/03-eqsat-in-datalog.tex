\chapter{Encoding E-graphs in Existing Datalog Systems}

In \autoref{chapter/nonrel-em},
 we discussed optimizations to make classical \ematching faster. 
As we see, there are still many limitations to the classical \ematching algorithm
 despite the proposed optimizations.
Query plans are limited to certain special forms,
 so many queries are asymptotically slower using classical \ematching.
Moreover,
 many advanced join algorithms (like the generic join algorithm) 
 and optimizations (like ones using cardinality estimation) cannot be used
 due to the fundamental restriction of its graph representation.
To enjoy the highly efficient \ematching procedure and the provided theoretical guarantees,
 we have to look back at the relational \ematching approach.
However,
 relational \ematching has the ``dual representation'' problem:
While \ematching is performed on the relaitonal representation,
 the graph representation is necessarily for standard \egraph operations
 like congruence maintenance.
Therefore,
 both representations are needed and should be kept in sync
 for relational \ematching to work.
This can have nontrivial overhead \citep{relational-ematching}.

A natural question to ask is, 
 if keeping both representation is expensive, 
 and efficient \ematching requires a relational representation,
 can we keep only the relational representation?
This way, 
 we are doing not only \ematching relationally,
 but also all other \egraph operations,
 and the ultimate goal is to be able to run equality saturation
 in this relational representation.
Compared to the optimizations described in \autoref{chapter/nonrel-em},
 this proposal is more radical,
 as it challenges the well-accepted assumption that an e-\textit{egraph} is a graph.
To implement this proposal, two key issues need to be addressed.
First,
 equality saturation uses equational rewrites to grow the \egraph,
 so it is important to understand the semantics of rewrites 
 in the relational representation.
Second,
 a key ingredient to \egraphs is the maintenance of its congruence invariant.
Therefore,
 a relational \egraph must be able to perform congruence maintenance as well.
To address the first issue, 
 we propose to encode \egraphs in Datalog.
Datalog is a relational language with rigourous semantics and efficient evaluation algorithms,
 where logic rules describe dependencies between relations.
Logic rules in Datalog have the form $R(\ldots) :- R1(\ldots),\ldots,Rn(\ldots)$ and
 operationally performs fixpoint-based rewrites but for relations.
Moreover, both rewrites in \egraphs and logics rules in Datalog are non-destructive,
 meaning that they do not remove original facts during the rewrites.
Therefore,
 it is tempting to encode \egraph rewrites in Datalog.

This chapter introduces my experience encoding \egraph rewrites 
 in two Datalog systems, namely Souffl\'e \citep{souffle} and Rel \citep{rel-doc}.
Souffl\'e and Rel are different in many aspects, with different targeted use cases:
 Souffl\'e focuses on applications like program analyses
 and has a semantics very similar to the original Datalog,
 with mild extensions like algebraic data types (ADTs),
 built-in support for equivalence relations, and the choice operator.
One of the most aggressive extension is perhaps
 the newly added subsumption,
 which allows the users to delete tuples 
 when it is clear that they are subsumed by other more general tuples \citep{datalog-subsumption}.
We will see subsumption is in fact critical in preventing the encoded \egraphs from blowup.
Rel, in contrast, has more ambitious goals.
While spiritually inspired by Datalog, 
 Rel has a much more expressive front end language based on first-order logic.
As an example, 
 queries in Rel support universal quantifiers and existential quantifiers in arbitrary positions
 (as long as the domain of the quantified variables are finitely enumerable).
Moreover,
 one important distinction between Souffl\'e and Rel is
 that Rel supports recursive aggregates out of box.
Rigourous theories are developed 
 for sound programming with recursive aggregates in Rel \citep{datalogo-convergence},
 yet to facilitate even more flexible general-purpose programming,
 soundness are not enforced in practice.
As a result,
 one needs to be careful when using recursive aggregates in Rel,
 to not violate properties like monotonicity.
I use recursive aggregates in both encodings:
 while Rel supports it out of box, 
 for Souffl\'e, I explicitly disabled the stratification checker.
Despite the wide use of recursive aggregates, the encoding is still sound,
 because it is semantically clear that rewrites in an \egraph is monotonic.
Moreover, in the encoding,
 tuples are only removed when they are subsumed by a more canonicalized version of them.

A key ingredient to making \egraph efficient is 
 to keep only the canonical tuples.
However, the encoding in both systems are not completely satisfying.
For Souffl\'e with the subsumption extension,
 a tuple can only be removed when it is able to find an evidence 
 that this tuple is subsumed.
For Rel, every iteration starts from scratch,
 so the only way to remove tuples is 
 to recompute all the facts in the current iteration while excluding the removed tuples,
 which is indirect.
Despite demonostrating the feasibility of encoding \egraphs in Datalog, 
 both encodings are practically very slow.
Constraint-Handlign Rules (CHR) \citep{chr} is a potential solution to this problem,
 as its rules allows more flexible removal of tuples.
Moreover,
 the literature has investigated 
 ways to encode the optimal implementation of union-find in CHR \citep{uf-chr},
 which is perhaps the most critical step in encoding an \egraph.
However, I did not pursue this approach for a long time, 
 since as far as I am aware, available implementations of CHR either misses important features,
 or is obsure and difficult to use.

Through out this chapter, we will use a very classical equality saturation program,
 namely associavity and communitativity of the $+$ operator, as our example.
The (pseudo)code in \autoref{fig:eqsat-example} shows how this can be defined in a library like \egg.
As a baseline, it takes less than one second for \egg to conclude that
 $\sum_{i=1\ldots 8}v_i$ is in the same \eclass as $\sum_{i=8\ldots 1}v_i$.
For our Datalog encoding,
 we did not expect it to be as efficient as highly specialized \egraph frameworks like \egg.
In fact, even the best encodings presented in this chapter
 are not capable of proving the above equivalence,
 although it is not unimaginable that a customized Datalog engine can be specialized
 for our \egraph encodings and therefore more efficient.
Moreover,
 for each of our encodings,
 it is either the case that there are more or less overheads 
 that will not been seen in a sensible \egraph impelementation,
 or we have to do some non-trivial hacking into the Datalog engine that 
 the engine impelementers will be surprised about.
Therefore, in some sense,
 our attempts to encode \egraphs in Datalog is unsatisfactory.
However, 
 as we will see,
 there are many joyful gems we will not be able to discover without these attempts.

\begin{figure}
\begin{lstlisting}[language=Rust, style=colouredRust]
// Enum declaration
define_language! {
    enum Expr {
        Add(Id, Id),
        Var(i64),
    }
}
// Rewrites
let rewrites = vec![
    rw!("(+ ?x ?y)" => "(+ ?y ?x)");
    rw!("(+ (+ ?x ?y) ?z) => "(+ ?x (+ ?y ?z))");
];
\end{lstlisting}
\caption{The example equality saturation program used in this chapter.}
\label{fig:eqsat-example}
\end{figure}

\section{Encoding E-graphs in Souffl\'e}

\subsection{Background}

Souffl\'e is a modern, efficient Datalog engine 
 that has wide applications in program analyses \citep{doop, souffle, souffle-interpreter}.
While sticking to the dogma of monotonicity, 
 Souffl\'e has been extended with a diverse range of extensions
 to both make it easier to program program analyses tasks
 and faster to run these tasks.
These extensions are amenable to the core theory of Datalog 
 (suppose the user does not break the assumptions)\footnote{With the exception
 of termination guarantees of pure Datalog. 
 Similar to programs in many other practical Datalog engines, 
 Souffl\'e programs may not terminate
 since they are allowed to populate new values, which is useful in practice.}.
We sketch some of these extensions that are used in our encoding below:

\subsubsection*{Algebraic Data Types}
Souffl\'e supports algebraic data types (ADTs) as columns.
For example, the program below below declares 
 an Abstract Syntax Tree of the example in \autoref{fig:eqsat-example}
 in Souffl\'e
 and populates the term $v_1+v_2$ in relation $R$:
\begin{verbatim}
    .type Id = Add {x : Id, y : Id}
        | Var {n : number}
    .decl R(Id).
    R($Add($Var(1), $Var(2))).
\end{verbatim}

Internally, Souffl\'e keeps a record table for ADTs, 
 where each tuple has a unique reference id, 
 the branch id for its constructor, and
 the field values.
Therefore, 
 the encoding is very similar to the one used 
 in relational e-matching, with the difference being
 in relational e-matching, different branches of an AST
 is represented as different tables, 
 not different ids within the same table.
This encoding allows Souffl\'e to 
 perform efficient join over ADTs.
The reader may wonder 
 why we still use ADTs while we can 
 simulate the same features with 
 the relational encoding 
 \textit{a la} the relational \ematching paper.
In fact,
 we use both:
 ADTs are specifically used in a skolemizing fashion,
 i.e., we use ADTs as a handy way to creating new \eclass ids.
For example, \verb|$Add(x, y)| represents the ``natural'' \eclass id
 of the \enode with symbol \verb|Add| and children $x$ and $y$.
Other approaches to creating new \eclass ids include 
 using the hash of the \enodes, which we used for Rel.

\def\eqrel{\texttt{eqrel}}

\subsubsection*{Equivalence relations}
Equivalence relations are widely used 
 for different program analyses tasks, 
 such as Bitcoin user group analysis \citep{anonymity-bitcoin} 
 and points-to analyses \citep{multi-alaias-analysis,points-to-linear}.
While directly writing these equivalence relations as
 transitive, reflexive, symmetric rules are highly inefficient,
 data structures like union find \citep{unionfind} can 
 make reasoning about equivalence orders of magnititude faster.
Motivated by this, Souffl\'e provides a built-in support for 
 equivalence relations named \eqrel. 
A relation declared as \eqrel{} will
 always satisfy the equivalence rules 
 and is implemented internally using union-find.
\eqrel{} is designed to be highly parallelizing, 
 and it compactly representes the equivalence relation
 in linear space, while it takes up to quadratic space
 to represent it directly.

\subsubsection*{Subsumptions}
Subsumption \citep{datalog-subsumption} is the idea that
 when one tuple is subsumed by another tuple semantically,
 it does not hurt to remove the subsumed tuples.
For example, 
 when computing the shortest paths between pairs of vertices in a graph,
 one may only care about the shortest paths. 
Consider the following Datalog program that computes the shortest path:
\begin{verbatim}
    p(x, y, c) :- e(x, y, c).
    p(x, y, c) :- p(x, z, cp), e(z, y, ce), c = cp + ce.
    sp(x, y, c) :- v(x), v(y), c = min c : p(x, y, c).
\end{verbatim}
This program will compute all possible paths between pairs of vertices,
 before aggregating over the paths to derive the shortest paths.
This is inefficient compared to the standard shortest path algorithms 
 like Dijkstra's algorithm.
Worst, when the graph contains (even positive) cycles, 
 these rules may not terminate, 
 because there are infinitely many paths.
Subsumption addresses this issue 
 by allowing the deletion of paths that are knwon to be not optimal,
 i.e., those non-shortest paths:
\begin{verbatim}
    sp(x, y, c) :- e(x, y, c).
    sp(x, y, c) :- sp(x, z, cp), e(z, y, ce), c = cp + ce.
    sp(x, y, c1) <= sp(x, y, c2) :- c1 >= c2.
\end{verbatim}
The last rule defines a partial order on
 \verb|sp| and says that tuple \verb|sp(x, y, c1)| 
 will be subsumed by tuple \verb|sp(x, y, c2)| if
 \verb|c1| is less than or equal to \verb|c2| 
 (note subsumption is a reflexive relation).
Operationally,
 the ``reduced set'' will be computed 
 after each iteration of evaluation according to
 the subsumptive rules.
\citet{datalog-subsumption} developed 
 a rigourous theory of subsumptions in Datalog
 and proved its soundness.
Finally,
 other approaches are proposed based on 
 semirings \citep{datalogo,datalogo-convergence}
 and lattices \citep{flix}.
For example, the Rel language,
 introduced in \autoref{section/rel},
 is based on the semiring approach.

In our encoding, we use subsumptions 
 to remove obsolete information.
For example,
 \eclasses are being constantly merged, updated, and canonicalized,
 which will cause \enodes to be canonicalized from time to time.
This leads to the existence of multiple representations of
 the same \enode, with only one being the canonical at any time.
Keeping these stale \enodes will explode the \egraph.
Instead,
 we can define a partial order over the \egraphs
 so that all stale \enodes are subsumed 
 by their canonical version and 
 let subsumptions to clean up the stale \enodes.
We will discuss this in details in \autoref{todo}.
 
\subsubsection*{User-defined functors}
While Souffl\'e provides a rich set of primitive operators,
 it further provides the flexibility by allowing the users
 to bring their own functions, which Souffl\'e calls
 user-defined functors.
To declare a user-defined functor, 
 the user defines its implementation in a C++ program and
 link it during the execution of the Souffl\'e program.
Some of the encodings use the user-defined functors 
 to make \eqrel{} more flexible \citep{zucker-udf-1,zucker-udf-2}.
Compared to the standard usage, 
 we use the user-defined functors in a rather wild way, 
 following \citep{zucker-udf-1} (\autoref{todo}).

\subsubsection*{Aggregations}
Finally, Souffl\'e supports stratified aggregations, 
 which is a standard extension to Datalog.
In other words, 
 Souffl\'e accepts programs where 
 aggregation operators like \texttt{max}, \texttt{min}, and \texttt{sum}
 does not transitively refer to themselves (i.e., are not recursive).
The stratification requirement is crucial to the soundness of the extension
 because it guarantees that the rules are monotonic.
Below is an example that does not satisfy the stratification:
\begin{verbatim}
    R(x) :- x = 1.
    R(c + 1) :- c = max x : R(x)
\end{verbatim}
Aftr the first iteration, the database $D_1$ will contain only $R(1)$. 
In the second iteration, because the second rule fires,
 the database $D_2$ will be $\{R(1), R(2)\}$.
However, in the third iteration,
 the application of the second rule to $D_2$ will yield
 $R(3)$, and $R(2)$ that used to exist in $D_2$ is now found nowhere,
 which breaks monotonicity.

That being said, there are Datalog programs that break monotonicity,
 yet are still monotonic, at least semantically (e.g., one with subsumptions).
Moreover, in our encoding, because it is semantically clear that 
 \egraphs are growing in a monotonic way, we use recursive aggregations 
 throughout.
Souffl\'e does not support recursive aggregations by default,
 so we pass the \verb|--disable-transformers=SemanticChecker|
 flag to Souffl\'e to disable the semantic check.
By doing this, 
 we entered the dangerous land of Souffl\'e 
 since all the assumptions checked by the semantic checker
 could be violated.
Moreover,
 since the design of Souffl\'e does not expect
 recursive use of aggregations,
 aggregations are computed using linear scan.
When aggregations are stratified,
 this is fine because all the aggregations are ``one-shot'',
 while when aggregations are used recursively, 
 this means that the aggregations
 are not maintained incrementally 
 and will repeatedly perform linear scans.
We half-address this issues with more hackings.

\section{Encoding E-graphs in Rel}\label{section/rel}

\section{Encoding E-graphs in }

thank Bernhard Scholz

thank Rel

thank Phil for his blog posts