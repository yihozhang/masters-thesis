\chapter{Encoding E-graphs in Existing Datalog Systems}

In \autoref{chapter/nonrel-em},
 we discussed optimizations to make classical \ematching faster. 
As we see, there are still many limitations to the classical \ematching algorithm
 despite the proposed optimizations.
Query plans are limited to certain special forms,
 so many queries are asymptotically slower using classical \ematching.
Moreover,
 many advanced join algorithms (like the generic join algorithm) 
 and optimizations (like ones using cardinality estimation) cannot be used
 due to the fundamental restriction of its graph representation.
To enjoy the highly efficient \ematching procedure and the provided theoretical guarantees,
 we have to look back at the relational \ematching approach.
However,
 relational \ematching has the ``dual representation'' problem:
While \ematching is performed on the relaitonal representation,
 the graph representation is necessarily for standard \egraph operations
 like congruence maintenance.
Therefore,
 both representations are needed and should be kept in sync
 for relational \ematching to work.
This can have nontrivial overhead \citep{relational-ematching}.

A natural question to ask is, 
 if keeping both representation is expensive, 
 and efficient \ematching requires a relational representation,
 can we keep only the relational representation?
This way, 
 we are doing not only \ematching relationally,
 but also all other \egraph operations,
 and the ultimate goal is to be able to run equality saturation
 in this relational representation.
Compared to the optimizations described in \autoref{chapter/nonrel-em},
 this proposal is more radical,
 as it challenges the well-accepted assumption that an e-\textit{egraph} is a graph.
To implement this proposal, two key issues need to be addressed.
First,
 equality saturation uses equational rewrites to grow the \egraph,
 so it is important to understand the semantics of rewrites 
 in the relational representation.
Second,
 a key ingredient to \egraphs is the maintenance of its congruence invariant.
Therefore,
 a relational \egraph must be able to perform congruence maintenance as well.
To address the first issue, 
 we propose to encode \egraphs in Datalog.
Datalog is a relational language with rigourous semantics and efficient evaluation algorithms,
 where logic rules describe dependencies between relations.
Logic rules in Datalog have the form $R(\ldots) :- R1(\ldots),\ldots,Rn(\ldots)$ and
 operationally performs fixpoint-based rewrites but for relations.
Moreover, both rewrites in \egraphs and logics rules in Datalog are non-destructive,
 meaning that they do not remove original facts during the rewrites.
Therefore,
 it is tempting to encode \egraph rewrites in Datalog.

This chapter introduces my experience encoding \egraph rewrites 
 in two Datalog systems, namely Souffl\'e \citep{souffle} and Rel \citep{rel-doc}.
Souffl\'e and Rel are different in many aspects, with different targeted use cases:
 Souffl\'e focuses on applications like program analyses
 and has a semantics very similar to the original Datalog,
 with mild extensions like algebraic data types (ADTs),
 built-in support for equivalence relations, and the choice operator.
One of the most aggressive extension is perhaps
 the newly added subsumption,
 which allows the users to delete tuples 
 when it is clear that they are subsumed by other more general tuples \citep{datalog-subsumption}.
We will see subsumption is in fact critical in preventing the encoded \egraphs from blowup.
Rel, in contrast, has more ambitious goals.
While spiritually inspired by Datalog, 
 Rel has a much more expressive front end language based on first-order logic.
As an example, 
 queries in Rel support universal quantifiers and existential quantifiers in arbitrary positions
 (as long as the domain of the quantified variables are finitely enumerable).
Moreover,
 one important distinction between Souffl\'e and Rel is
 that Rel supports recursive aggregates out of box.
Rigourous theories are developed 
 for sound programming with recursive aggregates in Rel \citep{datalogo-convergence},
 yet to facilitate even more flexible general-purpose programming,
 soundness are not enforced in practice.
As a result,
 one needs to be careful when using recursive aggregates in Rel,
 to not violate properties like monotonicity.
I use recursive aggregates in both encodings:
 while Rel supports it out of box, 
 for Souffl\'e, I explicitly disabled the stratification checker.
Despite the wide use of recursive aggregates, the encoding is still sound,
 because it is semantically clear that rewrites in an \egraph is monotonic.
Moreover, in the encoding,
 tuples are only removed when they are subsumed by a more canonicalized version of them.

A key ingredient to making \egraph efficient is 
 to keep only the canonical tuples.
However, the encoding in both systems are not completely satisfying.
For Souffl\'e with the subsumption extension,
 a tuple can only be removed when it is able to find an evidence 
 that this tuple is subsumed.
For Rel, every iteration starts from scratch,
 so the only way to remove tuples is 
 to recompute all the facts in the current iteration while excluding the removed tuples,
 which is indirect.
Despite demonostrating the feasibility of encoding \egraphs in Datalog, 
 both encodings are practically very slow.
Constraint-Handlign Rules (CHR) \citep{chr} is a potential solution to this problem,
 as its rules allows more flexible removal of tuples.
Moreover,
 the literature has investigated 
 ways to encode the optimal implementation of union-find in CHR \citep{uf-chr},
 which is perhaps the most critical step in encoding an \egraph.
However, I did not pursue this approach for a long time, 
 since as far as I am aware, available implementations of CHR either misses important features,
 or is obsure and difficult to use.

Through out this chapter, we will use a very classical equality saturation program,
 namely associavity and communitativity of the $+$ operator, as our example.
The (pseudo)code in \autoref{fig:eqsat-example} shows how this can be defined in a library like \egg.
As a baseline, it takes less than one second for \egg to conclude that
 $\sum_{i=1\ldots 8}v_i$ is in the same \eclass as $\sum_{i=8\ldots 1}v_i$.
For our Datalog encoding,
 we did not expect it to be as efficient as highly specialized \egraph frameworks like \egg.
In fact, even the best encodings presented in this chapter
 are not capable of proving the above equivalence,
 although it is not unimaginable that a customized Datalog engine can be specialized
 for our \egraph encodings and therefore more efficient.
Moreover,
 for each of our encodings,
 it is either the case that there are more or less overheads 
 that will not been seen in a sensible \egraph impelementation,
 or we have to do some non-trivial hacking into the Datalog engine that 
 the engine impelementers will be surprised about.
Therefore, in some sense,
 our attempts to encode \egraphs in Datalog is unsatisfactory.
However, 
 as we will see,
 there are many joyful gems we will not be able to discover without these attempts.

\begin{figure}
\begin{lstlisting}[language=Rust, style=colouredRust]
// Enum declaration
define_language! {
    enum Expr {
        Add(Id, Id),
        Var(i64),
    }
}
// Rewrites
let rewrites = vec![
    rw!("(+ ?x ?y)" => "(+ ?y ?x)");
    rw!("(+ (+ ?x ?y) ?z) => "(+ ?x (+ ?y ?z))");
];
\end{lstlisting}
\caption{The example equality saturation program used in this chapter.}
\label{fig:eqsat-example}
\end{figure}

\section{Encoding E-graphs in Souffl\'e}

\subsection{Background}

Souffl\'e is a modern, efficient Datalog engine 
 that has wide applications in program analyses \citep{doop, souffle, souffle-interpreter}.
While sticking to the dogma of monotonicity, 
 Souffl\'e has been extended with a diverse range of extensions
 to both make it easier to program program analyses tasks
 and faster to run these tasks.
These extensions are amenable to the core theory of Datalog 
 (suppose the user does not break the assumptions)\footnote{With the exception
 of termination guarantees of pure Datalog. 
 Similar to programs in many other practical Datalog engines, 
 Souffl\'e programs may not terminate
 since they are allowed to populate new values, which is useful in practice.}.
We sketch some of these extensions that are used in our encoding below:

\subsubsection*{Algebraic Data Types}
Souffl\'e supports algebraic data types (ADTs) as columns.
For example, the program below below declares 
 an Abstract Syntax Tree of the example in \autoref{fig:eqsat-example}
 in Souffl\'e
 and populates the term $v_1+v_2$ in relation $R$:
\begin{verbatim}
.type Id = Add {x : Id, y : Id}
    | Var {n : number}
.decl R(Id).
R($Add($Var(1), $Var(2))).
\end{verbatim}

Internally, Souffl\'e keeps a record table for ADTs, 
 where each tuple has a unique reference id, 
 the branch id for its constructor, and
 the field values.
Therefore, 
 the encoding is very similar to the one used 
 in relational e-matching, with the difference being
 in relational e-matching, different branches of an AST
 is represented as different tables, 
 not different ids within the same table.
This encoding allows Souffl\'e to 
 perform efficient join over ADTs.
The reader may wonder 
 why we still use ADTs while we can 
 simulate the same features with 
 the relational encoding 
 \textit{a la} the relational \ematching paper.
In fact,
 we use both:
 ADTs are specifically used in a skolemizing fashion,
 i.e., we use ADTs as a handy way to creating new \eclass ids.
For example, \verb|$Add(x, y)| represents the ``natural'' \eclass id
 of the \enode with symbol \verb|Add| and children $x$ and $y$.
Other approaches to creating new \eclass ids include 
 using the hash of the \enodes, which we used for Rel.

\def\eqrel{\texttt{eqrel}}

\subsubsection*{Equivalence relations}
Equivalence relations are widely used 
 for different program analyses tasks, 
 such as Bitcoin user group analysis \citep{anonymity-bitcoin} 
 and points-to analyses \citep{multi-alaias-analysis,points-to-linear}.
While directly writing these equivalence relations as
 transitive, reflexive, symmetric rules are highly inefficient,
 data structures like union find \citep{unionfind} can 
 make reasoning about equivalence orders of magnititude faster.
Motivated by this, Souffl\'e provides a built-in support for 
 equivalence relations named \eqrel. 
A relation declared as \eqrel{} will
 always satisfy the equivalence rules 
 and is implemented internally using union-find.
\eqrel{} is designed to be highly parallelizing, 
 and it compactly representes the equivalence relation
 in linear space, while it takes up to quadratic space
 to represent it directly.

- introduce features
    - ASTs
    - aggregates
    - equivalence relation
    - user-defined functors
    - subsumption




\section{Encoding E-graphs in Rel}

\section{Encoding E-graphs in }

thank Bernhard Scholz

thank Rel

thank Phil for his blog posts

thank Remy for his pure evil hack