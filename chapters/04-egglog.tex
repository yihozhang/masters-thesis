\chapter{Egglog}

This chapter presents an semi-formal on egglog, a made-up language that attempts to generalize
 both Datalog and egg.
Currently, only a partial implementation of the described egglog language exists, with some deviations.

Why egglog? 
The motivation behind egglog is to find a good model 
 for relational e-graphs that can take full advantage of 
 (1) performance of relational e-matching and 
 (2) expressiveness of Datalog, 
 while (3) being compatiable with egg as well as (4) efficient. 
This is the first approach described
at the beginning of \ref{chapter/nonrel-em}.
Compared to the second approach, 
 which as we saw is limited in the kind of optimization it can perform,
 this approach is more principled and fundamental.

egglog is a dialect of Datalog, so it supports various reasoning expressible in Datalog. 
A rule has the form \texttt{head1, ..., headn :- body1, ..., bodyn}.
For example, below is a valid egglog program:

\begin{verbatim}
rel link(string, string) from "./link.csv".
rel tc(string, string).

tc(a, b) :- link(a, b).
tc(a, b) :- link(a, c), tc(c, b).
\end{verbatim}

However, Datalog by itself is not that interesting.
So for the first part of the chapter,
 we will instead focus on the extensions 
 that make egglog interesting.
Next, we will give some examples and show why egglog generalizes egg.
We will also try to develop the operational and model semantics of egglog.

\section{Introduction to egglog}

\subsection{User-defined sorts and lattices}

In egglog, every value is either a (semi)lattice value or a sort value.
Lattices in egglog are algebraic structures 
 with a binary join operator ($\lor$)
 that is associative, commutative, and idempotent
 and a default top $\top$ where $\top\lor e=\top$ for all $e$.
For example, 
 standard types like \texttt{string}, \texttt{i64}, and \texttt{u64} in egglog are in fact 
 trivial lattices with $s_1\lor s_2 =\top$ for all $s_1\neq s_2$.
In egglog, $\top$ means unresolvable errors.
Users can define their own lattices by
 providing a definition for lattice join.

Similarly, users can define sorts. 
Unlike lattices, sorts are uninterpreted.
As a result, sort values can only be created implicitly 
 via functional dependency.
We will go back to this point later.

\subsection{Relations and Functional Dependencies}

Relations can be declared using the \texttt{rel} keyword. 
Moreover, it is possible to specify a functional dependency 
 between columns in egglog.
For example,
\begin{verbatim}
sort expr.
rel num(i64) -> expr.
\end{verbatim}
declares a sort called \texttt{expr} and
 a \texttt{num} relation with two columns \texttt{(i64, expr)}.
In the \texttt{num} relation, each \texttt{i64} uniquely determines the remaining column
 (i.e., \texttt{num(x, e1)} and \texttt{num(x, e2)} implies \texttt{e1 = e2}).
The \texttt{num} relation can be read as a function from \texttt{i64} to values in \texttt{expr}.
Similar declarations are ubiquitous in egglog to represent sort constructors.

As another example,
\begin{verbatim}
rel add(expr, expr) -> expr.
\end{verbatim}
declares a relation with three columns, 
and the first two columns together 
uniquely determines the third column.
This represents a constructor with two \texttt{expr} arguments.

Users can introduce new sort values with functional dependencies.
Example:
\begin{verbatim}
num(1, c). % equivalently, num(1, _).
num(2, d).
add(c, d, e) :- num(1, c), num(2, d).
\end{verbatim}

This program is interesting and 
 its semantics deviates from the one in standard Datalog.
In standard Datalog, this program will not compile 
 because variable \texttt{c} in the first rule, 
 \texttt{d} in the second rule,
 and \texttt{e} in the third rule are not bound.
However, this is a valid program in egglog.
Thanks to functional dependency, 
 variables in the head do not necessarily 
 have to be bound in the bodies.
Variables can be unbound as long as 
 they can be inferred from the functional dependency.
The above egglog program is roughly equivalent to the following Datalog program:
\begin{verbatim}
num(1, c) :- !num(1, _), c = new_expr().
num(2, d) :- !num(2, _), d = new_expr().
add(c, d, e) :- num(1, c), num(2, d), !add(c, d, _), e = new_expr()
\end{verbatim}
Negated atoms like \texttt{!num(1, \_)} is necessary here 
 because otherwise it will inserts more than one atoms matching \texttt{num(1, \_)},
 which violates the functional dependency associated to the relation.

The above example egglog program can also be written into one single rule with multiple heads:
\begin{verbatim}
add(c, d, e), num(1, c), num(2, d).
% roughly equivalent to 
% add(c, d, e), num(1, c), num(2, d) :- !num(1, _), !num(2, _), 
%                                       c = new_expr(), 
%                                       d = new_expr(),
%                                       !add(c, d, _),
%                                       e = new_expr().
\end{verbatim}

egglog also supports the bracket syntax, so the last program can be further simplified to:

\begin{verbatim}
add[num[1], num[2]].
\end{verbatim}

The bracket syntax will implicitly fill the omitted column(s) 
 with newly generated variable(s).
If the atom is nested within another term,
 the nested atom will be lifted to the top-level,
 and the generated variable(s) will take the original position of the atom.
Another silly example of the bracket syntax:

\begin{verbatim}
ans(x) :- xor[xor[x]].
% expands to
% ans(x) :- xor[y, z], xor(x, y, z)
% which expands to
% ans(x) :- xor(y, z, _), xor(x, y, z)
% this rule can be thought as
%   for any expr x, y where \texttt{y xor (x xor y)}
%   is present in the database, collect x as the result.
\end{verbatim}

Finally, in equational reasoning a la egg, 
 it is common to write rules like 
 "for every \texttt{(a + b) + c}, 
 populate \texttt{a + (b + c)} on the right
 and make them equivalent".
This rule will look like the following:
\begin{verbatim}
add(a, add[b, c], id) :- add(add[a, b], c, id).
\end{verbatim}

egglog further has a syntactic sugar for these equational rules:
 \texttt{head := body if body1 ... bodyn}
 where 
 both \texttt{head} and \texttt{body} should use the bracket syntax and omit the same number of columns.
The if clause can be omitted.
egglog will expand this syntactic sugar by 
 unfolding the top-level bracket in \texttt{head} and \texttt{body}
 with the same variable(s):
\begin{verbatim}
add[b, a] := add[a, b].
% unfolds to add(b, a, id) :- add(a, b, id).
add[a, add[b, c]] := add[add[a, b], c].
% unfolds to add(a, add[b, c], id) :- add(add[a, b], c, id).
num[1] := div[a, a] if num(x, a), x != 0.
% unfolds to num(1, id) :- div(a, a, id), num(x, a), x != 0.
\end{verbatim}

Note the equational rules may introduce functional dependency violation;
 for instance, last rule may cause multiple tuples to match \texttt{num(1, \_)},
 yet the first column should uniquely determines the tuple.
We will discuss more about how we resolve this kind of violations 
%  in the section on [Functional Dependency Repair](#ext-3-functional-dependency-repair).
The essential idea is that, if two sort values are present 
 with the same primary key, then the two sort values must be equivalent,
 whereas if two lattice values are present with the same primary key,
 the new, unique lattice value should generalize the two values, 
 i.e., it will be the least-upper bound of those lattice values.

% ### Relations with lattices

The example relations we see so far mostly center around sort values.
However, it is also possible and indeed very useful 
 to define relations with lattices:
\begin{verbatim}
rel hi(expr) -> lmax(-2147483648).
rel lo(expr) -> lmin(2147482647).
\end{verbatim}
To define a lattice column, 
 a default value need to be provided in the relation definition.
The default value is not a lattice bottom: 
 the bottom means do not exist.
Meanwhile, the lattice top means there are conflicts.
It is also possible for default value to refer to the determinant columns:

\begin{verbatim}
rel add1(i: i64) -> i64(i + 1).
\end{verbatim}

The column initialization syntax should be 
 reminiscent of C++'s 
 [member initializer lists](https://en.cppreference.com/w/cpp/language/constructor).

In the above example,
 \texttt{lo} and \texttt{hi} together define a range analysis for the \texttt{expr} sort.
This in facts generalizes the e-class analyses in egg.
Here are some rules for \texttt{hi} and \texttt{lo} (stolen from Zach):
\begin{verbatim}
hi(x, n.into()) :- num(n, x).
lo(x, n.into()) :- num(n, x).
lo(nx, n.negated()) :- hi(x, n), neg(x, nx).
hi(nx, n.negated()) :- lo(x, n), neg(x, nx).
lo(absx, 0) :- abs(x, absx).
lo[absx] := lo[x] if abs(x, absx), lo[x] >= 0.
hi[absx] := hi[x] if abs(x, absx), lo[x] >= 0.
lo(xy, lox + loy) :- lo(x, lox), hi(y, loy), add(x, y, xy).
% can be further simplified to
%   lo[xy] := lo[x] + lo[y] if add(x, y, xy)
\end{verbatim}

Note here instead of \texttt{lo(neg[x], n.negated()) :- hi(x, n).},
 we put the \texttt{neg} atom to the right-hand side and write
 \texttt{lo(nx, n.negated()) :- hi(x, n), neg(x, nx).}
There are some nuanced differences between the two rules.
This rule, besides doing what the second rule does, 
 always populates a \texttt{neg} tuple for each \texttt{hi} tuple even when it does not exist,
 so the first rule can be viewed as 
 an "annotation-only" version of the first rule,
 which is usually what we want.

The last example shows e-class analyses in egglog is composable
 (i.e., each analysis can freely refer to each other).
This is one of the reason why we believe egglog generalizes e-class analyses.
Moreover, they can also interact with other non-lattice relations in a meaningful way:\footnote{
The last rule in this example has a single variable on the left-hand side,
but the above mentioned syntactic expansion for \texttt{:=} does not apply to this case.
The rule is indeed equivalent to \texttt{abs(x, x) :- abs[x] if geq(x, num[0])}.
}
\begin{verbatim}
rel geq(expr, expr).

% ... some arithmetic rules ...

% need to convert to int because they are from different lattices
geq(a, b) :- lo[a].to_int() < hi[b].to_int().
% TODO: geq is quadratic in size. 
% egglog should support inlined relations
% to avoid materialize relations like geq

% ... other user-defined knowledge about geq

% x and abs[x] are equivalent when x > 0
x := abs[x] if geq(x, num[0])
\end{verbatim}

Diverging a little bit, 
 it is even possible to write the above rules without using analyses / relations with lattices:

\begin{verbatim}
sort bool.
rel true() -> bool.
rel false() -> bool.
rel geq(expr, expr) -> bool.

% for each abs[x] exists, populate geq[x, 0],
% in the hope that later 
% it will be "in the same e-class" as true[].
geq[x, 0] :- abs[x]

geq[numx, 0] := true[] if num(x, numx), x > 0.
geq[xx, 0] := true[] if mul(x, x, xx).
% if x > 0 and y > 0 are both equivalent to true,
% then x + y > 0 is also equivalent to true.
geq[xy, 0] := true[] 
           if add(x, y, xy), 
              geq(x, 0, true[]), 
              geq(y, 0, true[])
geq[xxyy, 0] := true[] if add(mul[x, x], mul[y, y], xxyy).

% ... other reasoning rules...

% if x>0 is equivalent to true,
% every abs[x] in the database should be equivalent to x 
x := abs[x] if geq(x, 0, true[])
\end{verbatim}

The above program can be seen as implementing
 a small theorem prover in egglog.
Whenever it sees abs[x], 
 a query about \texttt{x >= 0} will be issued to the database.
If later \texttt{x >= 0} is proven to be equivalent to true,
 a distinguished sort value,
 \texttt{abs[x]} will be put in the same e-class as \texttt{x}.

All these rewrite will be very hard to express in egg.

% ## Ext 3: Functional Dependency Repair

FDs can be violated: 
 what if the user introduced two values for the same set of determinant columns? 
In this case, we need to repair the FDs.
We have seen such examples many times in previous sections.
For example, rules like \texttt{R[x1, ..., xk] := ...} will add new values to \texttt{R} indexed by \texttt{x1, ..., xk},
 and it is likely that there are already other tuples with the same prefix \texttt{x1, ..., xk}.
These rules may potentially cause violation of functional dependencies.
In general, there are two kinds of violations:

**Case 1.** 
If the dependent column is a sort value, 
 egglog will unify the two sort values later in the iteration.
We can think of a term of a sort in egglog as a constant in some theories, 
 which refers to some element in the model. 
But we don’t know which element it refers to. 
However, by repairing functional dependencies, 
 we can get some clues about what the structure will look like.
Consider the following program
\begin{verbatim}
rel add(expr, expr) -> expr.
rel num(i64) -> expr.

% add the fact 2 + 1, where the last column is auto-generated.
add[num[2], num[1]]. 

% add the fact 2+1, but the last column is add[num[1], num[2]]
% (add[num[1], num[2]] is created on the fly because
% it occurs at the left hand side.)
add(num[2], num[1], 
    add[num[1], num[2]]). 
\end{verbatim}

Because now (without repairing) \texttt{add[num[1], num[2]]} will contain two rows. 
The functional dependency is violated.
If we think of rewriting under functional dependency as a process of finding a model for the sort,
 then what do we learn from this violation? 
We learned that, to respect the functional dependency,
 the two sort values must be the same thing!
Therefore 
 the expr originally referred by \texttt{add[num[2], num[1]]} and 
 by \texttt{add[num[1], num[2]]} will be treated as the same expr
 and no longer be distinguishable in egglog!
As we will show later, when a egglog program reaches the fixpoint, 
 it produces a valid, minimal model for the relations and the sorts 
 such that the rewrite rules and the functional dependencies are both respected.

TODO: mention no global union-find

**Case 2.**
What if the dependent column is a regular type as a Rust struct or an integer?
Well, we also need to unify them, but in a different way.
The idea here is to describe these values with a algebraic structure, 
 which in this case is a lattice. 
A lattice has a bottom (means does not exist) and a top (means conflicts).
Similar to [Flix](https://flix.dev/),
 lattice values will grow by taking the least upper bound of 
 all the violating tuples.
In that sense, egglog also generalizes Flix 
 (as is described in the [PLDI '16](https://dl.acm.org/doi/10.1145/2980983.2908096) paper).

% ## Ext 4: Seamless Interop with Rust

This proposed extension takes inspiration 
 from recent work on [Ascent](https://dl.acm.org/doi/pdf/10.1145/3497776.3517779),
 an expressive Datalog engine that has seamless integration
 with the Rust ecosystem.
One interesting feature of Ascent is that it allows 
 first-class introspection of the column values.
Ascent use this feature to support features like first-class environment
 (this and the next example are both from page 4 of the Ascent paper; comments are mine):
\begin{verbatim}
sigma(v, rho2, a, tick(e, t, k)) <--
  sigma(?e@Ref(x), rho, a, t), // the environment rho is enumerated here
  store(rho[x], ?Value(v, rho2)), // rho[x] is used as an index for store
  store(a, ?Kont(k));
\end{verbatim}

One thing though is that Ascent allows enumerating structs as a relation
 with the \texttt{for} keyword. For example:
\begin{verbatim}
sigma(v, rho2, store, a, tick(v, t,k)) <--
  sigma(?Ref(x), rho, store, a, t),
  // enumerating store[&rho[x]]
  for xv in store[&rho[x]].iter(), if let Value(v,rho2) = xv,
  // enumerating store[a]
  for av in store[a].iter(), if let Kont(k) = av;
\end{verbatim}
This makes Ascent have a more macro-y vibe,
 which makes sense since the whole Ascent frontend is based
 on Rust's procedural macros.
However, I think the similar can be easily achieved 
 inside the relational land,
 so in a full-fledged relational language like egglog,
 the \texttt{for} syntax may not be necessary.

Seamless interop with Rust is in general very powerful.
In fact, we have already used this feature a lot.
For example,
 lattices in egglog are structs defined in Rust 
 that implements certain traits.
So rules like \texttt{hi(x, n.into()) :- num(n, x).}, 
 will call methods 
 in the corresponding struct (e.g., \texttt{n.into()}).

In general, these user-defined functions 
 introduced functional dependencies from domains of functions to their range.
For example,
 rule \texttt{hi(x, n.into()) :- num(n, x).} can be viewed as
 \texttt{hi(x, n\_into) :- num(n, x), into\_rel(n, n\_into)} 
 with functional dependency from \texttt{n} to \texttt{n\_into}.
Advanced join algorithms
 like worst-case optimal joins
 can leverage these functional dependencies
 to optimize the query.
 
\section{The Model Semantics of egglog and its Evaluation}

In this section, we will focus on the problem of how to formalize egglog 
 and how to evaluate egglog programs. 
This section will first give the model semantics of egglog.
Then, It will describe
 rebuilding, an essential procedure for 
 evaluating and maintaining e-graphs,
 namely rebuilding, in the egglog setting.
Finally,
 we will discuss how egglog's matching procedure
 can benefit from semi-naive evaluation,
 a classic evaluation algorithm in Datalog

 \subsection{The Model Semantics}

For simplicity, we assume that 
 in our core language
 there will only be one (interpreted) lattice $L$
 and one (uninterpreted) sort $S$.

A relation declaration in core egglog has the following shape:

$$\text{rel }R(c_1,\dots,c_p)\rightarrow (s_1,\ldots, s_m, l_1,\ldots, l_n)$$
where $s_j$ is a sort value column, 
 $l_k$ is a lattice value column, 
 and $c_i$ can be either a lattice or sort value column. 
Such declaration specifies 
 a relation with schema  
 $(c_1, \ldots c_p, s_1, \ldots, s_m, l_1, \ldots, l_n)$
 and implies the following logical constraint:

$$
  \forall 
  \vec c,\vec s,\vec l,
  \vec{s'},\vec{l'}.
  R(\vec c, \vec s, \vec l)\land
  R(\vec{c'}, \vec{s'}, \vec{l'})\rightarrow
  \vec s=\vec{s'}\land \vec l = \vec{l'}$$
where $\vec x$ denotes a vector of variables $x_1,\ldots,x_k$.

Note in the core formalization, we don't assign default values to lattices 
and we assume the bottom is the default value, 
and each default value is specified as a rewrite rule.

A rewrite rule in the core looks like follows:
$$
  \exists \vec{z}.R_1(\vec{x_1}), \ldots, R_n(\vec {x_n})
  \gets
  S_1(\vec{y_1}), \ldots, S_m(\vec{y_m}).
$$
All variables $x_{ij}$ in the head are either bound in the body
 or existentially quantified.
Importantly, existentially quantified variables in the core
 quantifies over sort values and 
 must be "inferrable" from the functional dependency,
 meaning that they must be a dependent variable 
 within some relation atoms.
For example, the following egglog program is not valid,
 because rule \texttt{R(1, c)} can be triggered arbitrarily many times,
 each with different \texttt{c}:
\begin{verbatim}
sort S.
rel R(i64, S).
R(1, c). % translated from R[1].
\end{verbatim}

This "inferrable" constraints can be formalized as
$$\forall i.\exists j. z_i\in \text{dep}(R_j(\vec{x_j})),$$
where $\text{dep}\left(R_j\left(\vec{x_j}\right)\right)$ is the set of dependent variables in atom $R_j(\vec{x_j})$.

The rewrite rule in the core is further translated to the following logical constraint:

$$
  \forall \vec{y}.
  \left(\bigwedge_i \vec{y_i}\sqsubseteq_L S_i \rightarrow
  \exists \vec{z}\in S^k. \bigwedge_i \vec{x_i}\sqsubseteq_L R_i\right),$$
% <!-- \begin{verbatim}
% forall y11, ..., yn’mn’:
%   (/\i Si(yi1, ..., yim'i)) ->
%   exists z1 in L, ..., zk in L,
%     /\i (xi1, ... ximi) subset_L Ri
% \end{verbatim} -->
where $\vec{y}$ is the set of variables occurring in $\vec{y_1},\ldots, \vec{y_m}$ and
$$
(\vec{c},\vec{s},\vec{l})
\sqsubseteq_L R\iff
\exists \vec{l'}.
 R(\vec{c},\vec{s},\vec{l'}) \land\left(\bigwedge_i\;l_i\leq_L l'_i\right)
$$
% <!-- \end{verbatim}
% (s1, ..., sm, l1, ..., ln) subset_L R
% 	iff
% exists l'1, ..., l'n: 
%   R(s1, ..., sm, l'1, ..., l'n) /\
%   /\j (lj <= l'j)
% \end{verbatim} -->

In English, 
whenever a valuation of variables
 satisfied right-hand side, 
 there exists some $\vec{z}$ 
 such that the left-hand side also "holds",
 in the sense that some tuples in the relation
 subsumes the substituted left-hand side.

The result of evaluating a (core) egglog program is 
 the minimal model $(S_{\min}, D_{\min})$ that 
 satisfies the logical constraints derived from the program,
 where $S_{\min}$ is the minimal $S$ of sort values (up to isomorphism)
 and $D_{\min}$ is the minimal database instance (i.e., interpretation of relations) with domain $L$ and $S_{\min}$.

This core formalization should look familiar 
 to people who know the [chase](https://dl.acm.org/doi/10.1145/3034786.3034796): 
Functional dependencies are equality-generating dependencies (EGD),
 and rewrite rules are tuple-generating dependencies (TGD).
However, there are several critical distinctions between egglog and the chase.
First, the chase has both labelled nulls and constants, 
 and unifying two constants will cause a conflict.
Sort values in egglog can be thought as labelled nulls, 
 and there is no matching concept for constants in egglog.
Moreover,
 egglog supports lattice values.
This feature has its root in both egg's e-class analyses and relational languages like Flix,
 and is necessary for different kinds of analysis tasks
 that egglog strives to support.
Finally, and perhaps most importantly,
 egglog has this "inferrable" constraint for existential variables.
This constraint leads egglog 
 to have a single very efficient application algorithm
 of executing both EGDs (from functional dependencies) 
 and TGDs (from rewrite rules).
In contrast, 
 EGDs and TGDs in the chase are executed independently.
Without the lattice values and lifting the inferrable constraint,
 egglog programs can be expressed in the chase.

\subsection{The Evaluation Algorithm}

The evaluation algorithm of egglog programs 
 consists of two parts.
The core of the evaluation is the invariant-maintaining rebuilding algorithm,
 which is inspired 
 both by the rebuilding algorithm of egg and 
 by the evaluation algorithm of the chase.
The second part involves matching and applying egglog rules.
Applying egglog rules is efficient.
In the chase's terminology, 
 thanks to the above mentioned inferrable constraint, 
 rule application in egglog is able to utilize functional dependencies
 to avoid to generate unnecessary nulls.
Moreover,
 because egglog programs
 are monotonic computations over the relational database in nature,
 they can benefit from the semi-naive evaluation algorithm of Datalog.
We call this semi-naive matching, which can be seen as 
 a further improvement over
 [relational e-matching](https://dl.acm.org/doi/10.1145/3498696).


% ### Rebuilding

The rebuilding algorithm:

\begin{verbatim}
todo = mk_union_find()
domain = mk_set()

def union_sort(s1, s2):
  todo.union(s1, s2)
  domain.add_all([s1, s2])

def refresh_todo():
  todo = mk_union_find()
  domain = mk_set()

def on_insert(R, tup):
  # find the tuple by its determinant columns
  orig_tup = R.find_by_determinant(tup.det)
  if orig_tup is None:
    R.insert(tup)
  else:
    # enumerate each dependent column
    for c1 in tup.dep:
      col = c1.col
      c2 = orig_tup[col]
      if col.is_sort():
        s1 = todo.get_or_create(c1)
        s2 = todo.get_or_create(c2)
        union_sort(s1, s2)
      else:
        orig_tup.set_col(col, c1.lat_max(c2))

def normalize(tuple, union_find):
  return tuple.map(lambda val: 
    union_find.get_or_default(val, default = val))

def rebuild(DB):
  while not todo.is_empty():
    # take todo into the local scope
    union_find = todo
    refresh_todo()

    to_remove = mk_set()
    to_insert = mk_set()

    for val in domain:
      for R in DB:
        for col in R.cols:
          for tup in R.index_by(col = col, val = val):
            new_tup = normalize(tup, union_find)
            if new_tup != tup:
              to_remove.add((R, tup))
              to_insert.add((R, new_tup))

    DB.remove_all(to_remove)
    # may trigger on_insert
    DB.insert_all(to_insert)
\end{verbatim}

% ### Applying rewrite rules

\begin{verbatim}
def batch_rewrite(pats, DB):
  to_insert = mk_set()
  for (lhs, rhs) in pats:
    for subst in match(DB, lhs):
      subst = chase(DB, subst, lhs, rhs)
      for (R, atom) in rhs:
        to_insert.add((R, atom.apply(subst)))
  DB.insert_all(to_insert)
  return to_insert.is_empty()

def chase(DB, subst, lhs, rhs):
  shouldContinue = True
  while shouldContinue:
    shouldContinue = False

    for atom in rhs:
      det_vars = atom.get_det_vars()
      if det_vars.is_subset_of(subst.get_domain()):
        shouldContinue = True
        
        R = DB.get_rel(atom.rel)
        det = det_vars.apply(subst)
        tup = R.find_by_determinant(det)
        for var in atom.get_dep_vars():
          col = var.col
          if var.is_sort():
              if tup is None: continue
              value = tup.get_by_col(col)
              sort_update(subst, var, value)
          else:
            value = tup is None ? col.lat_init(det)
                                : tup.get_by_col(col)
            lat_update(subst, var, value)

  for var in rhs.get_all_vars():
    if !subst.contains(var):
      assert var.is_sort():
      subst[var] = new_sort_value(var.sort)

def lat_update(subst, var, value):
  if subst.contains(var):
    subst[var] = subst[var].lat_max(value)
  else:
    subst[var] = value

def sort_update(subst, var, value):
  if subst.contains(var):
    union_sort(subst[var], value)
  else:
    subst[var] = value
\end{verbatim}
% <!-- 
% Note to myself: 
% This part differs slightly from egglite.
% E.g., in egglite, a temp relation is built with all vars.
% Here we only accumulate a to_insert list.
% We can even stream process everything, so tuples to be
% inserted are inserted immediately.

% There's also a design choice about how to handle the case when
% multiple values exist for the same atom. 
% egglite will pick random one, which is correct because 
% at the end of the day they will be unified.
% However, is there a smarter way of doing this?
% Potentially in combination with how we do unify-- because
% there still seems to be some populate-then-unify redundancy.

% [BELOW ARE OBSOLETE]
% the application algorithm here is different from the one
% in egglite. In egglite, a temp relation is built.
% I think this is because the operator sqlite supports / the sql
% language is not rich enough so we can't stream process everything.
% Here we can.

% They also differ in how they handle the case when multiple
% values exist for the same atom (egglite will pick random one
% and wait the rebuilding to resolve the violation, egglog will
% ??? TODO), 
% -->

% ### The main algorithm

\begin{verbatim}python
def run(pats, DB, max_iter):
  for iter in range(max_iter):
    if !batch_rewrite(pats, DB):
      return
    rebuild(DB)
\end{verbatim}


% ### Semi-Naive Matching

One of the bottleneck in evaluating egglog programs
 is matching the left-hand side.
Since we are matching over a relational representation
 of the e-graphs,
 we are already doing is already relational e-matching.
However, we can go one step further:
Let \texttt{DB'} be the database of tuples 
 that are not touched in the current iteration of rewrite.
\texttt{DB'} by itself will not produce any interesting new tuples;
 it has to join with newly generated tuples (i.e., the delta database).
This is exactly the semi-naive evaluation algorithm of Datalog.
We call this similar optimization in egglog semi-naive matching.
This optimization will be tricky to do over e-graph's DAG representation,
 yet is fairly obvious in egglog's full-fledged relational representation.

\section{egglog by Example}

\subsection{Lambda Calculus}

\begin{verbatim}
sort term.
rel false() -> term.
rel true() -> term.
rel num(i32) -> term.
rel var(string) -> term.
rel add(term, term) -> term.
rel eq(term, term) -> term.
rel lam(string, term) -> term.
rel let(string, term, term) -> term.
rel fix(term, term) -> term.
rel cond(term, term, term) -> term.

rel free(term, string).
rel const_num(term, i32).
rel const_bool(term, bool).
rel is_const(term).

% constant folding
const_num(c, i) :- num(i, c).
const_num[c] := const_num[a] + const_num[b] 
             if c = add[a, b].
const_bool[true[], true].
const_bool[false[], false].
const_bool[c] := true[]
              if c = eq[a, a].
const_bool[c] := false[]
              if c = eq[a, b],
                 const_num[a] != const_num[b].
is_const(c) :- const_num[c].
is_const(c) :- const_bool[c].

% free variable analysis
free(c, v):- var(v, c).
free[c] := free[a] if c = add[a, b].
% unfolds to free(c, v) :- free(a, v), if c = add[a, b].
free[c] := free[b] if c = add[a, b].
free[c] := free[a] if c = eq[a, b].
free[c] := free[b] if c = eq[a, b].
free(c, v) :- free(body, v) 
           if c = lam[x, body],
              v != x.
% fv(let(x, a, b)) = free(a) + (free(b) \ x)
free(c, v) := free(b, v) 
           if c = let[x, a, b],
              v != x.
free[c] := free[a]
           if c = let[x, a, b].
free(c, v) :- free(body, v) 
           if c = fix[x, body],
              v != x.
free[c] := free[pred] if c = cond[pred, a, b]
free[c] := free[a] if c = cond[pred, a, b]
free[c] := free[b] if c = cond[pred, a, b]

% if-true
then := cond[true[], then, else].
% if-false
else := cond[false[], then, else].
% if-elim
else := cond[eq[var[x], e], then, else]
     if let[x, e, then] = let[x, e, else]
let[x, e, then] :- cond[eq[var[x], e], then, else]
let[x, e, else] :- cond[eq[var[x], e], then, else]
% add-comm
add[b, a] := add[a, b]
% add-assoc
add[a, add[b, c]] := add[add[a, b], c]
% eq-comm
eq[b, a] := eq[a, b]
% fix
let[v, fix[v, e], e] := fix[v, e]
% beta
let[v, e, body] := app[lam[v, body], e]
% let-app
app[let[v, e, a], let[v, e, b]] := let[v, e, app[a, b]]
% let-add
add[let[v, e, a], let[v, e, b]] := let[v, e, add[a, b]]
% let-eq
eq[let[v, e, a], let[v, e, b]] := let[v, e, eq[a, b]]
% let-const
c:= let[v, e, c] if is_const(c)
% let-if
cond[let[v, e, pred], let[v, e, then], let[v, e, else]] 
    := let[v, e, cond[pred, then, else]]
% let-var-same
e := let[v1, e, var[v1]]
% let-var-diff
var[v2] := let[v1, e, var[v2]] if v1 != v2
% let-lam-same
lam[v1, body] := let[v1, e, lam[v1, body]]
% let-lam-diff
lam[v2, let[v1, e, body]] := let[v1, e, lam[v2, body]] 
                          if v1 != v2, free(e, v2)
% capture-avoiding subst
lam[fresh, 
  let[v1, e, 
    let[v2, var[fresh], 
        body]]] 
  := 
let[v1, e, lam[v2, body]] 
  if v1 != v2, !free(e, v2), fresh = gensym().
\end{verbatim}

\subsection{Type Inference for HM Type System}

\begin{verbatim}
sort expr.
(decl-ctor lam (string expr) expr)
(decl-ctor var (string) expr)
(decl-ctor app (expr expr) expr)
(decl-ctor let (string expr) expr)
(decl-ctor unit () expr)

(decl-sort type)
(decl-ctor tvar (string) type)
(decl-ctor tunit () type)
(decl-ctor tarr (type type) type)
; we don't need to introduce additional
; rules for case like ((tvar s), tunit),
; where we will normally subst one to the other,
; because here once union! is performed,
; the object language can't distinguish
; (tvar s) from tunit
(rule (
    (eq (tarr fr1 to1) (tarr fr2 to2)
)(
    (union! fr1 fr2)
    (union! to1 to2)
))


; a ctx is a list of var -> scheme
(decl-sort ctx)
(decl-ctor nil () ctx)
(decl-ctor cons (string scheme) ctx)

; a scheme is a type with some type variables qualified
(decl-sort scheme)
(decl-ctor forall (string-list, type) scheme)


; \Sigma |- e : t
(decl-fun type-of 
    (: (ctx expr) type)
    (default \bot) ; in default case the type is None
                   ; each bottom value should be unique
                   ; so they don't accidentally merge
    (merge union!)) ; when merge, unify the type

(decl-fun generalize (: (type) scheme) TODO)
(decl-fun instantiate (: (scheme) type) TODO)

; \Sigma |- () : Unit
(merge (type-of ctx unit) tunit)

; \Sigma; x: t1 |- e : t2
; -------------------------------
; \Sigma |- lam x. e : t1 -> t2
(rule (
    (eq a (type-of ctx (lam x e)))
    (eq s (fresh-str))
)(
    (merge a (tarr (tvar s) 
                   (type-of (cons x (forall [] (tvar s)))
                               e)))
))

; \Sigma |- e1 :- t1 -> t2
; \Sigma |- e2 : t1
---------------------------------
; \Sigma |- e1 e2 : t2
(rule (
    (eq a (type-of ctx (app e1 e2)))
    (eq s (fresh-str))
)(
    (union! (type-of ctx e1) 
            (tarr (type-of ctx e2)
                  (tvar s)))
    (union! a (tvar s))
))


(rule (
    (eq a (type-of ctx (let x e1 e2)))
)(
    (union! a
            (type-of (cons x (generalize (type-of ctx e1)) 
                           ctx) 
                        e2))
))
\end{verbatim}

% # Comparison to other languages

% \subsection{Comparison to Rel}

% \subsection{Comparison to Souffle}

% \subsection{Comparison to Flix}

% \subsection{Comparison to the Chase}