
\documentclass [11pt, proquest] {uwthesis}[2020/02/24]
 
%
% The following line would print the thesis in a postscript font 

\usepackage{natbib}
\def\bibpreamble{\protect\addcontentsline{toc}{chapter}{Bibliography}}

\PassOptionsToPackage{unicode}{hyperref}
\PassOptionsToPackage{hyphens}{url}
\usepackage{amsmath,amssymb}
\usepackage{lmodern}
\usepackage{iftex}
\usepackage{hyperref}
\usepackage{url}
\usepackage{listings, listings-rust}
\usepackage{amsthm}
\usepackage{xcolor}
\IfFileExists{xurl.sty}{\usepackage{xurl}}{} % add URL line breaks if available
\IfFileExists{bookmark.sty}{\usepackage{bookmark}}{\usepackage{hyperref}}
% \hypersetup{
%   pdftitle={A Trick that Makes Classical E-Matching Faster},
%   pdfauthor={Yihong Zhang},
%   hidelinks}
\urlstyle{same} % disable monospaced font for URLs
\usepackage{color}
\usepackage{fancyvrb}
\newcommand{\VerbBar}{|}
\newcommand{\VERB}{\Verb[commandchars=\\\{\}]}
\DefineVerbatimEnvironment{Highlighting}{Verbatim}{commandchars=\\\{\}}

\newenvironment{Shaded}{}{}
\newcommand{\AlertTok}[1]{\textcolor[rgb]{1.00,0.00,0.00}{\textbf{#1}}}
\newcommand{\AnnotationTok}[1]{\textcolor[rgb]{0.38,0.63,0.69}{\textbf{\textit{#1}}}}
\newcommand{\AttributeTok}[1]{\textcolor[rgb]{0.49,0.56,0.16}{#1}}
\newcommand{\BaseNTok}[1]{\textcolor[rgb]{0.25,0.63,0.44}{#1}}
\newcommand{\BuiltInTok}[1]{#1}
\newcommand{\CharTok}[1]{\textcolor[rgb]{0.25,0.44,0.63}{#1}}
\newcommand{\CommentTok}[1]{\textcolor[rgb]{0.38,0.63,0.69}{\textit{#1}}}
\newcommand{\CommentVarTok}[1]{\textcolor[rgb]{0.38,0.63,0.69}{\textbf{\textit{#1}}}}
\newcommand{\ConstantTok}[1]{\textcolor[rgb]{0.53,0.00,0.00}{#1}}
\newcommand{\ControlFlowTok}[1]{\textcolor[rgb]{0.00,0.44,0.13}{\textbf{#1}}}
\newcommand{\DataTypeTok}[1]{\textcolor[rgb]{0.56,0.13,0.00}{#1}}
\newcommand{\DecValTok}[1]{\textcolor[rgb]{0.25,0.63,0.44}{#1}}
\newcommand{\DocumentationTok}[1]{\textcolor[rgb]{0.73,0.13,0.13}{\textit{#1}}}
\newcommand{\ErrorTok}[1]{\textcolor[rgb]{1.00,0.00,0.00}{\textbf{#1}}}
\newcommand{\ExtensionTok}[1]{#1}
\newcommand{\FloatTok}[1]{\textcolor[rgb]{0.25,0.63,0.44}{#1}}
\newcommand{\FunctionTok}[1]{\textcolor[rgb]{0.02,0.16,0.49}{#1}}
\newcommand{\ImportTok}[1]{#1}
\newcommand{\InformationTok}[1]{\textcolor[rgb]{0.38,0.63,0.69}{\textbf{\textit{#1}}}}
\newcommand{\KeywordTok}[1]{\textcolor[rgb]{0.00,0.44,0.13}{\textbf{#1}}}
\newcommand{\NormalTok}[1]{#1}
\newcommand{\OperatorTok}[1]{\textcolor[rgb]{0.40,0.40,0.40}{#1}}
\newcommand{\OtherTok}[1]{\textcolor[rgb]{0.00,0.44,0.13}{#1}}
\newcommand{\PreprocessorTok}[1]{\textcolor[rgb]{0.74,0.48,0.00}{#1}}
\newcommand{\RegionMarkerTok}[1]{#1}
\newcommand{\SpecialCharTok}[1]{\textcolor[rgb]{0.25,0.44,0.63}{#1}}
\newcommand{\SpecialStringTok}[1]{\textcolor[rgb]{0.73,0.40,0.53}{#1}}
\newcommand{\StringTok}[1]{\textcolor[rgb]{0.25,0.44,0.63}{#1}}
\newcommand{\VariableTok}[1]{\textcolor[rgb]{0.10,0.09,0.49}{#1}}
\newcommand{\VerbatimStringTok}[1]{\textcolor[rgb]{0.25,0.44,0.63}{#1}}
\newcommand{\WarningTok}[1]{\textcolor[rgb]{0.38,0.63,0.69}{\textbf{\textit{#1}}}}
\setlength{\emergencystretch}{3em} % prevent overfull lines
\providecommand{\tightlist}{%
  \setlength{\itemsep}{0pt}\setlength{\parskip}{0pt}}

\newtheorem{example}{Example}

\usepackage{stmaryrd}
\setcounter{tocdepth}{1}  % Print the chapter and sections to the toc

\renewcommand{\chapterautorefname}{Chapter}
\renewcommand{\subsubsectionautorefname}{Section}
\renewcommand{\subsectionautorefname}{Section}
\renewcommand{\sectionautorefname}{Section}

\usepackage{xspace}

\newcommand{\GJ}{\textsf{GJ}\xspace}
\newcommand{\EM}{\textsf{EM}\xspace}

\newcommand{\sys}{\texttt{Qry}\xspace}
\newcommand{\Egraph}{\mbox{E-graph}\xspace}
\newcommand{\egraph}{\mbox{e-graph}\xspace}
\newcommand{\Egraphs}{\mbox{E-graphs}\xspace}
\newcommand{\egraphs}{\mbox{e-graphs}\xspace}
\newcommand{\Eclass}{\mbox{E-class}\xspace}
\newcommand{\eclass}{\mbox{e-class}\xspace}
\newcommand{\Eclasses}{\mbox{E-classes}\xspace}
\newcommand{\eclasses}{\mbox{e-classes}\xspace}
\newcommand{\ematch}{\mbox{e-match}\xspace}
\newcommand{\ematching}{\mbox{e-matching}\xspace}
\newcommand{\Ematching}{\mbox{E-matching}\xspace}
\newcommand{\Enodes}{\mbox{E-nodes}\xspace}
\newcommand{\enodes}{\mbox{e-nodes}\xspace}
\newcommand{\enode}{\mbox{e-node}\xspace}
\newcommand{\equivid}{\ensuremath{\equiv_{\sf id}}\xspace}
\newcommand{\find}{\textsf{find}\xspace}
\newcommand{\lookup}{\textsf{lookup}\xspace}
\newcommand{\egg}{\texttt{egg}\xspace}

\begin{document}

\prelimpages

\Title{Towards a Relational E-graph}
\Author{Yihong Zhang}
\Year{2022}
\Program{Computer Science \& Engineering}
\Programtext{}
\textofChair{}
\Signature{Professor Zachary Tatlock}

\copyrightpage
\titlepage  

% \setcounter{page}{-1}
\abstract{%
This thesis presents my experience improving the performance and expressiveness of \egraphs
 with both relational and non-relational approaches.
\autoref{chapter/nonrel-em} presents a non-relational optimization algorithm for \ematching
 and studied its properties.
Motivated by the insufficiency of the non-relational approach, 
 we turn to study ways to encode \egraphs inside \autoref{chapter/datalog}.
We also describe a prototype language for relational \egraphs 
 and its semantics (\autoref{chapter/egglog}).
\autoref{chapter/related-works} gives a survey of related works.
}
 
\tableofcontents
% \listoffigures
% \listoftables  % I have no tables

\acknowledgments{\vskip2pc
  {\narrower\noindent
  This thesis summarizes the work done during my master
  on building and understanding relational \egraphs.
  It builds upon previous work on relational \ematching done during my undergrad.
  Both my master and undergrad are done at UW,
  and after my master's journey, 
  I will work as a PhD student, also at UW, to continue the work 
  presented in this thesis.

  I want to thank my advisor Zachary Tatlock for his guidance 
  during the development of this thesis and for his mentorship over the past years.
  I would also like to thank my collaborators for many insightful discussions.
  Among them are Max Willsey, Remy Wang, Philip Zucker, Eli Rosenthal.
  Finally, I want to thank my family and friends 
  for their support.
  \par}
}

\dedication{\begin{center}to Jiawei\end{center}}

\textpages
 
% \chapter {Introduction}
 

\section{The Purpose of This Sample Thesis}

\chapter{Optimizing Non-relational E-matching}\label{chapter/nonrel-em}

E-matching is an important procedure for many \egraph based applications,
 yet it is slow.
In a typical application of equality saturation,
 \ematching can take 60-90\% of the overall run time \citep{egg}.
In the work presented in my bachelor's thesis, 
 my collaborators and I proposed a relational approach to \ematching, 
 dubbed relational \ematching \citep{relational-ematching, relational-ematching-thesis}.
In particular,
 we made e-matching orders of magnitude faster, 
 proved theoretical bounds of e-matching, 
 and opened the door for all kinds of
 wild optimizations that can be done with databases and e-graphs.

% I'm very proud of
% this work, not only because it is elegant, fast, and theoretically
% (worst-case) optimal, but also because it is the kind of work I'd like
% to work on: building connections across areas.

However, 
 the relational e-matching approach also has some secret pitfalls.
In particular, 
 to have the best of both efficient e-graph maintenance and efficient e-matching, 
 one has to switch back and forth 
 between the e-graph to its relational representation.
Our prototype\footnote{\url{https://github.com/egraphs-good/egg/tree/relational}}
 builds a relational database and associated indices from
 scratch for each match-apply iteration.
This is acceptable in the equality saturation setting.
E-matching and updates always alternate in batches, 
 so the cost of building the database is amortized.
Plus, 
 since building databases and indices are both linear time costs, 
 they are often subsumed by the time spent on e-matching.

However, 
 what if e-matching is not run in batches? 
Or what if all the e-matching patterns are quite simple 
 and the constant overhead is now a bottleneck? 
An \egraph framework can implement some fast paths for that, 
 but then there are more design questions to consider: 
Are we going to keep two implementations of e-matching? 
What kind of queries should be computed by relational
 e-matching and what by traditional e-matching? \ldots{} 
We can continue down this path and put a lot of engineering effort into
 building a practically efficient e-graph engine, or we can:

\begin{enumerate}
\item
  Start a clean-slate relational e-graph framework that handles all
  e-graph operations efficiently and forget about the graph part of an
  e-graph;
\item
  Keep the current \egraph data structure, and port some optimizations of
  relational e-matching back to \egraphs.
\end{enumerate}

In this chapter,
 I will focus on the second approach.
The rest of the thesis will focus on the first approach.
When working on relational e-matching, 
 we found an optimization to the backtracking-style classical e-matching.
Like relational e-matching, 
 it is able to improve e-matching asymptotically in some cases, 
 but it does not require transforming the input \egraph to a relational database.
And it is very simple.
For what it is worth, 
 this optimization (instead of relational e-matching) is what is currently
 implemented in egg.

In this chapter, I will describe this optimization.
But before that,
 we will go over some brief background on e-matching.
Readers are welcome to skip it if they already know what e-matching is.

\section{E-Matching}\label{e-matching}

There have been many great introductions to e-graphs and e-matching.
For example, 
 Philip Zucker gives
 a gentle introduction to 
 \egraphs \citep{zucker-egraph-1} and
 \ematching \citep{zucker-egraph-2} in Julia.
Max Willsey also wrote a very nice tutorial \citep{egg-tutorial}
 on e-graphs and egg \citep{egg}.
Basically, 
 an e-graph is a data structure that compactly represents 
 an equivalence relation and 
 e-matching is pattern matching
 on such e-graphs modulo equivalence.
Both e-matching and e-graphs are widely used in
 SMT solvers \citep{efficient-ematching}
 and equality saturation-based
 program optimizers \citep{tensat}.
A typical equality saturation-based program optimizer may
 take the majority of its time doing e-matching.

There are several algorithms proposed for e-matching.
For example,
 the one currently used in egg is based
 on the virtual machine proposed by \citet{efficient-ematching}.
The traditional backtracking-based e-matching algorithm
 does not exploit equality constraints during pattern compilation.
Equality constraints are the term we used in the relational e-matching paper
 to describe the kind of constraints that all occurrences of the same variables should be mapped
 to equivalent terms.
Those that violate the equality constraints will
 not be pruned away immediately.
For example, \(f(\alpha, g(\alpha))\)
 does not match \(f(1,g(2))\), 
 because the first \(\alpha\) is mapped to
 $1$ but the second is mapped to $2$.
The classical backtracking-based
 e-matching will still consider it though.

The relational e-matching approach instead treats an e-matching pattern
 as a kind of relational query.
From a relational query, 
 the query optimizer can easily identify all kinds of constraints, 
 including equality constraints,
 and find an efficient query plan.
As an example,
 the above pattern can be compiled to query
 \(Q(r, \alpha)\gets R_f(r, \alpha,x),R_g(x,\alpha)\), 
 and a hash join could answer this query in linear time.

\section{The optimization}\label{the-optimization}

The issue with the traditional backtracking-style e-matching 
 is that it does not take advantage of the equality constraints, 
 so it enumerates obviously unsatisfying terms.
The optimization is therefore straightforward: 
 do not enumerate terms that are obviously unsatisfying.
And this is easy, because we already know what the (only) satisfying
 term should look like!

Let us first look at the classical e-matching algorithm 
 (reproduced from \citet[Figure~3]{relational-ematching}, with a typo fix).

\begin{align*}
    \textit{match}(x,c,S) = 
                   & \{ \sigma \cup \{ x \mapsto c\} \mid \sigma \in S, x \not \in \text{dom}(\sigma)\}\ \cup\\
                   & \{ \sigma \mid \sigma \in S, \sigma(x) = c \}\\
    \textit{match}(f(p_{1}, \dots, p_{k}), c, S) = 
                   & \bigcup_{f(c_{1},\dots,c_{k})\in c}
                     \textit{match}(p_{k}, c_{k}, \dots, \textit{match}(p_{1}, c_{1}, S))
\end{align*}

It takes a pattern \(p\), an e-class \(c\), 
 and current substitutions \(S\),
 and returns the set of substitutions produced by e-matching \(p\)
 over e-class \(c\),
 such that all produced substitutions are extensions
 of some substitutions in \(S\).
The result of e-matching a pattern \(p\)
 over an e-graph is
 \(\bigcup_{c\in C} \textit{match}(p, c, \{\emptyset\})\)
 (both \citet{relational-ematching} and \citet{efficient-ematching}
 have another typo here), where \(C\) is the
 set of e-classes in the e-graph.

The algorithm is straightforward:
\begin{enumerate}
\tightlist
\item
  If the pattern is a variable, and
  \begin{enumerate}
  \tightlist
  \item
    if this variable is fresh in the domain of the substitution, then
    it is safe to extend the substitutions with \(\{x\mapsto c\}\), or
  \item
    if this variable is not fresh, we keep only these substitutions that
    are consistent with the mapping \(\{x\mapsto c\}\).
  \end{enumerate}
\item
  If the pattern is a function symbol of the form \(f(p_1,\ldots,p_k)\),
  the algorithm iterates over \(f\)-nodes \(f(c_1,\ldots, c_k)\) in the
  e-class, and fold over the sub patterns and sub e-classes with
  \(\textit{match}\), to accumulate set of valid substitutions.
\end{enumerate}

The trick is to generalize case 1.b.
In case 1.b, we know the
 substitution for a pattern is unique when the pattern is a non-fresh
 variable, but we \emph{also} know this when the variables of the pattern
 are in the domain of the substitution (i.e.,
 \(\text{fv}(p)\subseteq\text{dom}(S)\)), thanks to canonicalization.
In that case, 
 the pattern after substitution is a ground term, 
 which can be efficiently looked up in a bottom-up fashion.

To implement this idea, we lift case 1.b to the top-level of the
 algorithm.
During e-matching, 
 the algorithm will first check whether the
 free vars of the input pattern is contained in the domain of the
 substitution.
If yes, 
 then instead of looking further into the pattern,
 the algorithm will lookup the substituted term for comparison.
The following definition shows this:
\begin{align*}
    \textit{match}(p, c, S) = & \begin{cases}
        \{\sigma \mid \sigma\in S, 
                      \textit{lookup}([\sigma]e)=c\} 
        &\text{ if $\text{fv}(p)\subseteq \text{dom}(S)$ }\\
        match'(p, c, S)&\text{ o.w.}
    \end{cases}\\
    \textit{match'}(x,c,S) = 
                   & \{ \sigma \cup \{ x \mapsto c\} \mid \sigma \in S\}\\
    \textit{match'}(f(p_{1}, \dots, p_{k}), c, S) = 
                   & \bigcup_{f(c_{1},\dots,c_{k})\in c}
                     \textit{match}(p_{k}, c_{k}, \dots, \textit{match}(p_{1}, c_{1}, S))
\end{align*}
In the above definition, we also drop the check of
\(x\not\in\text{dom}(\sigma)\) for the variable case, which is
guaranteed not to happen.

As an example, consider \(f(\alpha, g(\alpha))\) again.
E-matching will
 enumerate through each \(f\)-node and bind \(\alpha\) to the first child
 of the \(f\)-node.
Here, the classical e-matching algorithm will then
 enumerate though the second child e-class of the \(f\)-node for possible
 \(g\)-nodes.
However, because \(g(\alpha)\) is a ground term after
 substituting \(\alpha\) with \(\sigma(\alpha)\), we can effectively
 lookup \(g(\alpha)\) and compare it with the e-class id of the second
 child.
The pseudocode:
\begin{Shaded}
\begin{Highlighting}[]
\CommentTok{\# classical e{-}matching}
\ControlFlowTok{for}\NormalTok{ f }\KeywordTok{in}\NormalTok{ c: }\CommentTok{\# f(a, g(a))}
  \ControlFlowTok{for}\NormalTok{ g }\KeywordTok{in}\NormalTok{ f.child2: }\CommentTok{\# g(a)}
    \ControlFlowTok{if}\NormalTok{ f.child1 }\OperatorTok{!=}\NormalTok{ g.child1:}
      \ControlFlowTok{continue}
    \ControlFlowTok{yield}\NormalTok{ \{a: f.child1\}}

\CommentTok{\# with the trick}
\ControlFlowTok{for}\NormalTok{ f }\KeywordTok{in}\NormalTok{ c: }\CommentTok{\# f(a, g(a))}
\NormalTok{  g }\OperatorTok{=}\NormalTok{ lookup(mk\_node(g, f.child1))}
  \ControlFlowTok{if}\NormalTok{ g }\KeywordTok{is} \VariableTok{None} \KeywordTok{or}\NormalTok{ g }\OperatorTok{!=}\NormalTok{ f.child2:}
    \ControlFlowTok{continue}
  \ControlFlowTok{yield}\NormalTok{ \{a: f.child1\}}
\end{Highlighting}
\end{Shaded}

Implementation-wise, egg adds a new operator to the e-matching virtual
 machine called \texttt{Lookup}.
\texttt{Lookup} (1) substitutes the
 pattern with values in the VM register to produce a ground term and (2)
 lookup the ground term in the e-graph.

\section{A Relational View of the Trick}\label{a-relational-view-of-the-trick}

How effective is this trick? To have a better understanding of this
 trick, we need to take a relational lens.
The classical e-matching can be
 viewed as a relational query plan where hash joins only index one column
 (the link between parent and child) and potentially prune using the rest
 of equality columns (the equality constraint).
At first I thought this
 optimization will make classical e-matching equivalent to some efficient
 hash join-based query plans, 
 and a efficient plan here specifically means a plan
 where the hash joins will index all the columns known to be equivalent.
But this is false.
Consider the pattern \(f(\alpha, g(\alpha,\beta))\).
The relational version of it is
 \(Q(r, \alpha,\beta)\gets R_f(r, \alpha, x),R_g(x,\alpha,\beta)\).
An efficient plan with hash joins will index both \(\alpha\) and \(x\).
However,
 our trick cannot use the \(\alpha\) in \(f\) to prune the
 \(\alpha\) in \(g\), because there could be multiple satisfying
 \(g\)-nodes (due to the unbound variable \(\beta\)).
In this case, our
 optimization does not offer any speedup.

In fact, this trick can be relationally thought of as the kind of query
 optimizations that leverage functional dependencies.
In the relational representation of e-graphs,
 there is a functional dependency from the
 children columns to the id column.
For example, in relation
 \(R_f(x, c_1, c_2)\), the relational representation of binary function
 symbol \(f\), every combination of \(c_1\) and \(c_2\) uniquely
 determines \(x\) thanks to e-graph canonicalization.
Our trick uses this
 information to immediately determine the value of \(x\) once \(c_1\) and
 \(c_2\) are known, without looking at obviously unsatisfying candidates.

In the relational e-matching paper,
 we also described how we use
 functional dependency to speed up queries.
In fact, if the variable
 ordering of generic joins follows the topological order of the (acyclic)
 functional dependency,
 the run-time complexity will be worst-case
 optimal \textit{under the presence of FDs} \citep{wcoj-survey}, 
 a stronger guarantee than the original
 AGM bound \citep{agm}.
Functional dependencies are also exploited
 for query optimization in databases \citep{data-dep-for-opt-survey}.

How does this compare to relational e-matching? 
First, as we saw above,
 it is not as powerful as relational e-matching.
Moreover, 
 the graph representation has the fundamental restriction 
 that makes it very hard
 to do advanced optimizations, 
 e.g., one that uses cardinality information.
It is also limited in the kind of join it is able to
 (conceptually) perform (only hash joins).
However, 
 it integrates well
 with an existing non-relational e-graph framework,
 which relational \ematching fails to achieve.

\section{Query planning}\label{query-planning}

This trick also poses a new question for classical e-matching planning:
what visit order should one use? In the above definition of our algorithm,
 we assumed a depth-first style order of processing, but this is not
 necessary.
For example,
 after enumerating the top-level \(f\)-node in
 pattern \(f(g(\alpha), h(\alpha, \beta))\), it will be most efficient to
 enumerate the \(h\)-node and lookup \([\sigma]g(\alpha)\) later.
If however we first enumerate \(g(\alpha)\),
we still cannot avoid enumerating \(h(\alpha,\beta)\) later on.

If we assume the cost of enumerating each node is the same,
 this problem can be viewed as finding the smallest connected component (CC) in the
 pattern tree that contains the root, such that the CC covers all
 distinct variables.
This is not an easy problem, and similar problems are NP-hard.
This problem can be solved 
 using dynamic programming on trees with exponential states, or can be reduced
 to an ILP problem.
However, both seem to be an overkill for realistic queries.

It is also unclear what is a practically good planning heuristic.
The one used in egg prioritizes sub-patterns with more free vars,
 but this may not be good enough: 
 consider pattern
 \(f(f(g(\alpha),\beta)),g(h(\alpha), h(\beta)))\).
 This heuristics yield the following plan for this pattern:

\begin{Shaded}
\begin{Highlighting}[]
\ControlFlowTok{for}\NormalTok{ f1 }\KeywordTok{in}\NormalTok{ c: }\CommentTok{\# f(f(g(a), g(b))), g(h(a), h(b)))}
  \ControlFlowTok{for}\NormalTok{ f2 }\KeywordTok{in}\NormalTok{ c.child1: }\CommentTok{\# f(g(a), g(b))) (2 free vars)}
    \ControlFlowTok{for}\NormalTok{ g1 }\KeywordTok{in}\NormalTok{ c.child2: }\CommentTok{\# g(h(a), h(b)) (2 free vars)}
      \ControlFlowTok{for}\NormalTok{ g2 }\KeywordTok{in}\NormalTok{ f2.child1: }\CommentTok{\# g(a) (1 free var)}
\NormalTok{        h1 }\OperatorTok{=}\NormalTok{ lookup(mk\_node(h, g2.child1) }\CommentTok{\# lookup h(a)}
        \ControlFlowTok{if}\NormalTok{ h1 }\KeywordTok{is} \VariableTok{None} \KeywordTok{or}\NormalTok{ h1 }\OperatorTok{!=}\NormalTok{ g1.child1:}
          \ControlFlowTok{continue}
        \ControlFlowTok{for}\NormalTok{ g3 }\KeywordTok{in}\NormalTok{ f2.child2: }\CommentTok{\# g(b) (1 free var)}
\NormalTok{          h2 }\OperatorTok{=}\NormalTok{ lookup(mk\_node(h, g3.child1) }\CommentTok{\# lookup h(b)}
          \ControlFlowTok{if}\NormalTok{ h2 }\KeywordTok{is} \VariableTok{None} \KeywordTok{or}\NormalTok{ h2 }\OperatorTok{!=}\NormalTok{ g1.child2:}
            \ControlFlowTok{continue}
          \ControlFlowTok{yield}\NormalTok{ \{...\}}
\end{Highlighting}
\end{Shaded}

This is complicated, but it suffices to only look at the first three loops: 
 It does a cross product over the first and the second child of
 the top-level \(f\)-node.
A good strategy here is instead to prefer
 fewer free vars, and performs the search in a depth-first search, 
 so that \(g(h(a), h(b))\) can be looked up all at once after
 \(f(g(a), g(b))\) is enumerated.
But it is not yet known if preferring fewer free vars is the
 right strategy.
Moreover, realistic patterns
 tend to be small and simple, 
 so cases like the above may be rare.

\section{Miscellaneous}\label{miscellaneous}

This chapter is adapted 
 from my blog post 
 \textit{A Trick that Makes Classical E-Matching Faster} \citep{nonrelational-ematching-post}.
I thank Max and Philip for their valuable discussions and comments.
The presented trick stems from a Pull Request\footnote{\url{https://github.com/egraphs-good/egg/pull/74}}
 that tries to improve e-matching for ground terms.
In hindsight, 
 a variant of the proposed improvement targeting multi-patterns
 had been discussed in \citet{efficient-ematching}
 but was lost in egg's original e-matching implementation.
Compared to that Pull Request, which
 only looks up ground terms, this optimization generalizes it by also
 looking up terms that are grounded after substitution.
Philip came up with this idea independently 
 as well\footnote{\url{https://github.com/egraphs-good/egg/pull/74\#issuecomment-818833367}}.

\chapter{Encoding E-graphs in Existing Datalog Systems}

In \autoref{chapter/nonrel-em},
 we discussed optimizations to make classical \ematching faster. 
As we see, there are still many limitations to the classical \ematching algorithm
 despite the proposed optimizations.
Query plans are limited to certain special forms,
 so many queries are asymptotically slower using classical \ematching.
Moreover,
 many advanced join algorithms (like the generic join algorithm) 
 and optimizations (like ones using cardinality estimation) cannot be used
 due to the fundamental restriction of its graph representation.
To enjoy the highly efficient \ematching procedure and the provided theoretical guarantees,
 we have to look back at the relational \ematching approach.
However,
 relational \ematching has the ``dual representation'' problem:
While \ematching is performed on the relaitonal representation,
 the graph representation is necessarily for standard \egraph operations
 like congruence maintenance.
Therefore,
 both representations are needed and should be kept in sync
 for relational \ematching to work.
This can have nontrivial overhead \citep{relational-ematching}.

A natural question to ask is, 
 if keeping both representation is expensive, 
 and efficient \ematching requires a relational representation,
 can we keep only the relational representation?
This way, 
 we are doing not only \ematching relationally,
 but also all other \egraph operations,
 and the ultimate goal is to be able to run equality saturation
 in this relational representation.
Compared to the optimizations described in \autoref{chapter/nonrel-em},
 this proposal is more radical,
 as it challenges the well-accepted assumption that an e-\textit{egraph} is a graph.
To implement this proposal, two key issues need to be addressed.
First,
 equality saturation uses equational rewrites to grow the \egraph,
 so it is important to understand the semantics of rewrites 
 in the relational representation.
Second,
 a key ingredient to \egraphs is the maintenance of its congruence invariant.
Therefore,
 a relational \egraph must be able to perform congruence maintenance as well.
To address the first issue, 
 we propose to encode \egraphs in Datalog.
Datalog is a relational language with rigourous semantics and efficient evaluation algorithms,
 where logic rules describe dependencies between relations.
Logic rules in Datalog have the form $R(\ldots) :- R1(\ldots),\ldots,Rn(\ldots)$ and
 operationally performs fixpoint-based rewrites but for relations.
Moreover, both rewrites in \egraphs and logics rules in Datalog are non-destructive,
 meaning that they do not remove original facts during the rewrites.
Therefore,
 it is tempting to encode \egraph rewrites in Datalog.

This chapter introduces my experience encoding \egraph rewrites 
 in two Datalog systems, namely Souffl\'e \citep{souffle} and Rel \citep{rel-doc}.
Souffl\'e and Rel are different in many aspects, with different targeted use cases:
 Souffl\'e focuses on applications like program analyses
 and has a semantics very similar to the original Datalog,
 with mild extensions like algebraic data types (ADTs),
 built-in support for equivalence relations, and the choice operator.
One of the most aggressive extension is perhaps
 the newly added subsumption,
 which allows the users to delete tuples 
 when it is clear that they are subsumed by other more general tuples \citep{datalog-subsumption}.
We will see subsumption is in fact critical in preventing the encoded \egraphs from blowup.
Rel, in contrast, has more ambitious goals.
While spiritually inspired by Datalog, 
 Rel has a much more expressive front end language based on first-order logic.
As an example, 
 queries in Rel support universal quantifiers and existential quantifiers in arbitrary positions
 (as long as the domain of the quantified variables are finitely enumerable).
Moreover,
 one important distinction between Souffl\'e and Rel is
 that Rel supports recursive aggregates out of box.
Rigourous theories are developed 
 for sound programming with recursive aggregates in Rel \citep{datalogo-convergence},
 yet to facilitate even more flexible general-purpose programming,
 soundness are not enforced in practice.
As a result,
 one needs to be careful when using recursive aggregates in Rel,
 to not violate properties like monotonicity.
I use recursive aggregates in both encodings:
 while Rel supports it out of box, 
 for Souffl\'e, I explicitly disabled the stratification checker.
Despite the wide use of recursive aggregates, the encoding is still sound,
 because it is semantically clear that rewrites in an \egraph is monotonic.
Moreover, in the encoding,
 tuples are only removed when they are subsumed by a more canonicalized version of them.

A key ingredient to making \egraph efficient is 
 to keep only the canonical tuples.
However, the encoding in both systems are not completely satisfying.
For Souffl\'e with the subsumption extension,
 a tuple can only be removed when it is able to find an evidence 
 that this tuple is subsumed.
For Rel, every iteration starts from scratch,
 so the only way to remove tuples is 
 to recompute all the facts in the current iteration while excluding the removed tuples,
 which is indirect.
Despite demonostrating the feasibility of encoding \egraphs in Datalog, 
 both encodings are practically very slow.
Constraint-Handlign Rules (CHR) \citep{chr} is a potential solution to this problem,
 as its rules allows more flexible removal of tuples.
Moreover,
 the literature has investigated 
 ways to encode the optimal implementation of union-find in CHR \citep{uf-chr},
 which is perhaps the most critical step in encoding an \egraph.
However, I did not pursue this approach for a long time, 
 since as far as I am aware, available implementations of CHR either misses important features,
 or is obsure and difficult to use.

Through out this chapter, we will use a very classical equality saturation program,
 namely associavity and communitativity of the $+$ operator, as our example.
The (pseudo)code in \autoref{fig:eqsat-example} shows how this can be defined in a library like \egg.
As a baseline, it takes less than one second for \egg to conclude that
 $\sum_{i=1\ldots 8}v_i$ is in the same \eclass as $\sum_{i=8\ldots 1}v_i$.
For our Datalog encoding,
 we did not expect it to be as efficient as highly specialized \egraph frameworks like \egg.
In fact, even the best encodings presented in this chapter
 are not capable of proving the above equivalence,
 although it is not unimaginable that a customized Datalog engine can be specialized
 for our \egraph encodings and therefore more efficient.
Moreover,
 for each of our encodings,
 it is either the case that there are more or less overheads 
 that will not been seen in a sensible \egraph impelementation,
 or we have to do some non-trivial hacking into the Datalog engine that 
 the engine impelementers will be surprised about.
Therefore, in some sense,
 our attempts to encode \egraphs in Datalog is unsatisfactory.
However, 
 as we will see,
 there are many joyful gems we will not be able to discover without these attempts.

\begin{figure}
\begin{lstlisting}[language=Rust, style=colouredRust]
// Enum declaration
define_language! {
    enum Expr {
        Add(Id, Id),
        Var(i64),
    }
}
// Rewrites
let rewrites = vec![
    rw!("(+ ?x ?y)" => "(+ ?y ?x)");
    rw!("(+ (+ ?x ?y) ?z) => "(+ ?x (+ ?y ?z))");
];
\end{lstlisting}
\caption{The example equality saturation program used in this chapter.}
\label{fig:eqsat-example}
\end{figure}

\section{Encoding E-graphs in Souffl\'e}

\subsection{Background}

Souffl\'e is a modern, efficient Datalog engine 
 that has wide applications in program analyses \citep{doop, souffle, souffle-interpreter}.
While sticking to the dogma of monotonicity, 
 Souffl\'e has been extended with a diverse range of extensions
 to both make it easier to program program analyses tasks
 and faster to run these tasks.
These extensions are amenable to the core theory of Datalog 
 (suppose the user does not break the assumptions)\footnote{With the exception
 of termination guarantees of pure Datalog. 
 Similar to programs in many other practical Datalog engines, 
 Souffl\'e programs may not terminate
 since they are allowed to populate new values, which is useful in practice.}.
We sketch some of these extensions that are used in our encoding below:

\subsubsection*{Algebraic Data Types}
Souffl\'e supports algebraic data types (ADTs) as columns.
For example, the program below below declares 
 an Abstract Syntax Tree of the example in \autoref{fig:eqsat-example}
 in Souffl\'e
 and populates the term $v_1+v_2$ in relation $R$:
\begin{verbatim}
    .type Id = Add {x : Id, y : Id}
        | Var {n : number}
    .decl R(Id).
    R($Add($Var(1), $Var(2))).
\end{verbatim}

Internally, Souffl\'e keeps a record table for ADTs, 
 where each tuple has a unique reference id, 
 the branch id for its constructor, and
 the field values.
Therefore, 
 the encoding is very similar to the one used 
 in relational e-matching, with the difference being
 in relational e-matching, different branches of an AST
 is represented as different tables, 
 not different ids within the same table.
This encoding allows Souffl\'e to 
 perform efficient join over ADTs.
The reader may wonder 
 why we still use ADTs while we can 
 simulate the same features with 
 the relational encoding 
 \textit{a la} the relational \ematching paper.
In fact,
 we use both:
 ADTs are specifically used in a skolemizing fashion,
 i.e., we use ADTs as a handy way to creating new \eclass ids.
For example, \verb|$Add(x, y)| represents the ``natural'' \eclass id
 of the \enode with symbol \verb|Add| and children $x$ and $y$.
Other approaches to creating new \eclass ids include 
 using the hash of the \enodes, which we used for Rel.

\def\eqrel{\texttt{eqrel}}

\subsubsection*{Equivalence relations}
Equivalence relations are widely used 
 for different program analyses tasks, 
 such as Bitcoin user group analysis \citep{anonymity-bitcoin} 
 and points-to analyses \citep{multi-alaias-analysis,points-to-linear}.
While directly writing these equivalence relations as
 transitive, reflexive, symmetric rules are highly inefficient,
 data structures like union find \citep{unionfind} can 
 make reasoning about equivalence orders of magnititude faster.
Motivated by this, Souffl\'e provides a built-in support for 
 equivalence relations named \eqrel. 
A relation declared as \eqrel{} will
 always satisfy the equivalence rules 
 and is implemented internally using union-find.
\eqrel{} is designed to be highly parallelizing, 
 and it compactly representes the equivalence relation
 in linear space, while it takes up to quadratic space
 to represent it directly.

\subsubsection*{Subsumptions}
Subsumption \citep{datalog-subsumption} is the idea that
 when one tuple is subsumed by another tuple semantically,
 it does not hurt to remove the subsumed tuples.
For example, 
 when computing the shortest paths between pairs of vertices in a graph,
 one may only care about the shortest paths. 
Consider the following Datalog program that computes the shortest path:
\begin{verbatim}
    p(x, y, c) :- e(x, y, c).
    p(x, y, c) :- p(x, z, cp), e(z, y, ce), c = cp + ce.
    sp(x, y, c) :- v(x), v(y), c = min c : p(x, y, c).
\end{verbatim}
This program will compute all possible paths between pairs of vertices,
 before aggregating over the paths to derive the shortest paths.
This is inefficient compared to the standard shortest path algorithms 
 like Dijkstra's algorithm.
Worst, when the graph contains (even positive) cycles, 
 these rules may not terminate, 
 because there are infinitely many paths.
Subsumption addresses this issue 
 by allowing the deletion of paths that are knwon to be not optimal,
 i.e., those non-shortest paths:
\begin{verbatim}
    sp(x, y, c) :- e(x, y, c).
    sp(x, y, c) :- sp(x, z, cp), e(z, y, ce), c = cp + ce.
    sp(x, y, c1) <= sp(x, y, c2) :- c1 >= c2.
\end{verbatim}
The last rule defines a partial order on
 \verb|sp| and says that tuple \verb|sp(x, y, c1)| 
 will be subsumed by tuple \verb|sp(x, y, c2)| if
 \verb|c1| is less than or equal to \verb|c2| 
 (note subsumption is a reflexive relation).
Operationally,
 the ``reduced set'' will be computed 
 after each iteration of evaluation according to
 the subsumptive rules.
\citet{datalog-subsumption} developed 
 a rigourous theory of subsumptions in Datalog
 and proved its soundness.
Finally,
 other approaches are proposed based on 
 semirings \citep{datalogo,datalogo-convergence}
 and lattices \citep{flix}.
For example, the Rel language,
 introduced in \autoref{section/rel},
 is based on the semiring approach.

In our encoding, we use subsumptions 
 to remove obsolete information.
For example,
 \eclasses are being constantly merged, updated, and canonicalized,
 which will cause \enodes to be canonicalized from time to time.
This leads to the existence of multiple representations of
 the same \enode, with only one being the canonical at any time.
Keeping these stale \enodes will explode the \egraph.
Instead,
 we can define a partial order over the \egraphs
 so that all stale \enodes are subsumed 
 by their canonical version and 
 let subsumptions to clean up the stale \enodes.
We will discuss this in details in \autoref{todo}.
 
\subsubsection*{User-defined functors}
While Souffl\'e provides a rich set of primitive operators,
 it further provides the flexibility by allowing the users
 to bring their own functions, which Souffl\'e calls
 user-defined functors.
To declare a user-defined functor, 
 the user defines its implementation in a C++ program and
 link it during the execution of the Souffl\'e program.
Some of the encodings use the user-defined functors 
 to make \eqrel{} more flexible \citep{zucker-udf-1,zucker-udf-2}.
Compared to the standard usage, 
 we use the user-defined functors in a rather wild way, 
 following \citep{zucker-udf-1} (\autoref{todo}).

\subsubsection*{Aggregations}
Finally, Souffl\'e supports stratified aggregations, 
 which is a standard extension to Datalog.
In other words, 
 Souffl\'e accepts programs where 
 aggregation operators like \texttt{max}, \texttt{min}, and \texttt{sum}
 does not transitively refer to themselves (i.e., are not recursive).
The stratification requirement is crucial to the soundness of the extension
 because it guarantees that the rules are monotonic.
Below is an example that does not satisfy the stratification:
\begin{verbatim}
    R(x) :- x = 1.
    R(c + 1) :- c = max x : R(x)
\end{verbatim}
Aftr the first iteration, the database $D_1$ will contain only $R(1)$. 
In the second iteration, because the second rule fires,
 the database $D_2$ will be $\{R(1), R(2)\}$.
However, in the third iteration,
 the application of the second rule to $D_2$ will yield
 $R(3)$, and $R(2)$ that used to exist in $D_2$ is now found nowhere,
 which breaks monotonicity.

That being said, there are Datalog programs that break monotonicity,
 yet are still monotonic, at least semantically (e.g., one with subsumptions).
Moreover, in our encoding, because it is semantically clear that 
 \egraphs are growing in a monotonic way, we use recursive aggregations 
 throughout.
Souffl\'e does not support recursive aggregations by default,
 so we pass the \verb|--disable-transformers=SemanticChecker|
 flag to Souffl\'e to disable the semantic check.
By doing this, 
 we entered the dangerous land of Souffl\'e 
 since all the assumptions checked by the semantic checker
 could be violated.
Moreover,
 since the design of Souffl\'e does not expect
 recursive use of aggregations,
 aggregations are computed using linear scan.
When aggregations are stratified,
 this is fine because all the aggregations are ``one-shot'',
 while when aggregations are used recursively, 
 this means that the aggregations
 are maintained non-incrementally 
 and will repeatedly perform linear scans.
This can be prohibitively expensive, 
 and we half-address this issues with more hackings.

\section{Encoding E-graphs in Rel}\label{section/rel}

\section{Encoding E-graphs in }

thank Bernhard Scholz

thank Rel

thank Phil for his blog posts
% egg-lite

Welcome to the tutorial on Gogi (short for eg**g**l**ogi**sh), a made-up language that attempts to generalize
both [Datalog](https://en.wikipedia.org/wiki/Datalog) 
and [egg](https://egraphs-good.github.io).
This blog stems from
a [trick](https://github.com/nikomatsakis/plmw-2022) I learned from Nicholas Matsakis at PLMW 2022:
To write a tutorial for a non-existing language.
By doing this, I can get a sense of what I want from this new language as well as early feedbacks from others.

Why Gogi? 
The motivation behind Gogi is to find a good model 
for relational e-graphs that can take full advantage of 
(1) performance of relational e-matching and 
(2) expressiveness of Datalog, 
while (3) being compatiable with egg as well as (4) efficient. 
This is the first approach I described 
at the beginning of the previous [post](blog/ematch-trick.html). 
I'm actually more excited about this approach, 
because I believe this is _the right way_ in long term.

Gogi is Datalog, so it supports various reasoning expressible in Datalog. 
A rule has the form `head1, ..., headn :- body1, ..., bodyn`.
For example, below is a valid Gogi program:

```prolog
rel link(string, string) from "./link.csv".
rel tc(string, string).

tc(a, b) :- link(a, b).
tc(a, b) :- link(a, c), tc(c, b).
```

TODO:
However, Datalog by itself is not that interesting.
So for the first part of the post,
I will instead focus on the extensions 
that make Gogi interesting.
Next, I'll give some examples and show why Gogi generalizes egg
I will also try to develop the operational and model semantics of Gogi.
<!-- Finally, I'll discuss some thoughts on Gogi. -->


# Introduction to Gogi

## Ext 1: User-defined sorts and lattices

In Gogi, every value is either a (semi)lattice value or a sort value.
Lattices in Gogi are algebraic structures 
with a binary join operator ($\lor$)
that is associative, commutative, and idempotent
and a default top $\top$ where $\top\lor e=\top$ for all $e$.
For example, 
standard types like `string`, `i64`, and `u64` in Gogi are in fact 
trivial lattices with $s_1\lor s_2 =\top$ for all $s_1\neq s_2$.
In Gogi, $\top$ means unresolvable errors.
Users can define their own lattices by
providing a definition for lattice join.

Similarly, users can define sorts. 
Unlike lattices, sorts are uninterpreted.
As a result, sort values can only be created implicitly 
via functional dependency.
We will go back to this point later.

## Ext 2: Relations and Functional Dependencies

### Declaring a relation with functional dependency

Relations can be declared using the `rel` keyword. 
Moreover, it is possible to specify a functional dependency 
between columns in Gogi.
For example,
```prolog
sort expr.
rel num(i64) -> expr.
```
declares a sort called `expr` and
a `num` relation with two columns `(i64, expr)`.
In the `num` relation, each `i64` uniquely determines the remaining column
(i.e., `num(x, e1)` and `num(x, e2)` implies `e1 = e2`).
The `num` relation can be read as a function from `i64` to values in `expr`.
Similar declarations are ubiquitous in Gogi to represent sort constructors.

As another example,
```prolog
rel add(expr, expr) -> expr.
```
declares a relation with three columns, 
and the first two columns together 
uniquely determines the third column.
This represents a constructor with two `expr` arguments.

Users can introduce new sort values with functional dependencies.
Example:
```prolog
num(1, c). % equivalently, num(1, _).
num(2, d).
add(c, d, e) :- num(1, c), num(2, d).
```

This program is interesting and 
its semantics deviates from the one in standard Datalog.
In standard Datalog, this program will not compile 
because variable `c` in the first rule, 
`d` in the second rule,
and `e` in the third rule are not bound.
However, this is a valid program in Gogi.
Thanks to functional dependency, 
variables in the head do not necessarily 
have to be bound in the bodies.
Variables can be unbound as long as 
they can be inferred from the functional dependency.
The above Gogi program is roughly equivalent to the following Datalog program:
```prolog
num(1, c) :- !num(1, _), c = new_expr().
num(2, d) :- !num(2, _), d = new_expr().
add(c, d, e) :- num(1, c), num(2, d), !add(c, d, _), e = new_expr()
```
Negated atoms like `!num(1, _)` is necessary here 
because otherwise it will inserts more than one atoms matching `num(1, _)`,
which violates the functional dependency associated to the relation.

The above example Gogi program can also be written into one single rule with multiple heads:
```prolog
add(c, d, e), num(1, c), num(2, d).
% roughly equivalent to 
% add(c, d, e), num(1, c), num(2, d) :- !num(1, _), !num(2, _), 
%                                       c = new_expr(), 
%                                       d = new_expr(),
%                                       !add(c, d, _),
%                                       e = new_expr().
```

### The bracket syntax

Gogi also supports the bracket syntax, so the last program can be further simplified to:

```prolog
add[num[1], num[2]].
```

The bracket syntax will implicitly fill the omitted column(s) 
with newly generated variable(s).
If the atom is nested within another term,
the nested atom will be lifted to the top-level,
and the generated variable(s) will take the original position of the atom.
Another silly example of the bracket syntax:

```prolog
ans(x) :- xor[xor[x]].
% expands to
% ans(x) :- xor[y, z], xor(x, y, z)
% which expands to
% ans(x) :- xor(y, z, _), xor(x, y, z)
% this rule can be thought as
%   for any expr x, y where `y xor (x xor y)`
%   is present in the database, collect x as the result.
```

Finally, in equational reasoning a la egg, 
it is common to write rules like 
"for every `(a + b) + c`, 
populate `a + (b + c)` on the right
and make them equivalent".
This rule will look like the following:
```prolog
add(a, add[b, c], id) :- add(add[a, b], c, id).
```

Gogi further has a syntactic sugar for these equational rules:
`head := body if body1 ... bodyn`
where 
both `head` and `body` should use the bracket syntax and omit the same number of columns.
The if clause can be omitted.
Gogi will expand this syntactic sugar by 
unfolding the top-level bracket in `head` and `body`
with the same variable(s):
```prolog
add[b, a] := add[a, b].
% unfolds to add(b, a, id) :- add(a, b, id).
add[a, add[b, c]] := add[add[a, b], c].
% unfolds to add(a, add[b, c], id) :- add(add[a, b], c, id).
num[1] := div[a, a] if num(x, a), x != 0.
% unfolds to num(1, id) :- div(a, a, id), num(x, a), x != 0.
```

Note the equational rules may introduce functional dependency violation;
for instance, last rule may cause multiple tuples to match `num(1, _)`,
yet the first column should uniquely determines the tuple.
We will discuss more about how we resolve this kind of violations 
in the section on [Functional Dependency Repair](#ext-3-functional-dependency-repair).
The essential idea is that, if two sort values are present 
with the same primary key, then the two sort values must be equivalent,
whereas if two lattice values are present with the same primary key,
the new, unique lattice value should generalize the two values, 
i.e., it will be the least-upper bound of those lattice values.

### Relations with lattices

The example relations we see so far mostly center around sort values.
However, it is also possible and indeed very useful 
to define relations with lattices:
```prolog
rel hi(expr) -> lmax(-2147483648).
rel lo(expr) -> lmin(2147482647).
```
To define a lattice column, 
a default value need to be provided in the relation definition.
The default value is not a lattice bottom: 
the bottom means do not exist.
Meanwhile, the lattice top means there are conflicts.
It is also possible for default value to refer to the determinant columns:

```prolog
rel add1(i: i64) -> i64(i + 1).
```

The column initialization syntax should be 
reminiscent of C++'s 
[member initializer lists](https://en.cppreference.com/w/cpp/language/constructor).

In the above example,
`lo` and `hi` together define a range analysis for the `expr` sort.
This in facts generalizes the e-class analyses in egg.
Here are some rules for `hi` and `lo` (stolen from Zach):
```prolog
hi(x, n.into()) :- num(n, x).
lo(x, n.into()) :- num(n, x).
lo(nx, n.negated()) :- hi(x, n), neg(x, nx).
hi(nx, n.negated()) :- lo(x, n), neg(x, nx).
lo(absx, 0) :- abs(x, absx).
lo[absx] := lo[x] if abs(x, absx), lo[x] >= 0.
hi[absx] := hi[x] if abs(x, absx), lo[x] >= 0.
lo(xy, lox + loy) :- lo(x, lox), hi(y, loy), add(x, y, xy).
% can be further simplified to
%   lo[xy] := lo[x] + lo[y] if add(x, y, xy)
```

Note here instead of `lo(neg[x], n.negated()) :- hi(x, n).`,
we put the `neg` atom to the right-hand side and write
`lo(nx, n.negated()) :- hi(x, n), neg(x, nx).`
There are some nuanced differences between the two rules.
This rule, besides doing what the second rule does, 
always populates a `neg` tuple for each `hi` tuple even when it does not exist,
so the first rule can be viewed as 
an "annotation-only" version of the first rule,
which is usually what we want.

The last example shows e-class analyses in Gogi is composable
(i.e., each analysis can freely refer to each other).
This is one of the reason why we believe Gogi generalizes e-class analyses.
Moreover, they can also interact with other non-lattice relations in a meaningful way:
^[
   The last rule in this example has a single variable on the left-hand side,
   but the above mentioned syntactic expansion for `:=` does not apply to this case.
   The rule is indeed equivalent to `abs(x, x) :- abs[x] if geq(x, num[0])`.
]
```prolog
rel geq(expr, expr).

% ... some arithmetic rules ...

% need to convert to int because they are from different lattices
geq(a, b) :- lo[a].to_int() < hi[b].to_int().
% TODO: geq is quadratic in size. 
% Gogi should support inlined relations
% to avoid materialize relations like geq

% ... other user-defined knowledge about geq

% x and abs[x] are equivalent when x > 0
x := abs[x] if geq(x, num[0])
```

Diverging a little bit, 
it is even possible to write the above rules without using analyses / relations with lattices:

```prolog
sort bool.
rel true() -> bool.
rel false() -> bool.
rel geq(expr, expr) -> bool.

% for each abs[x] exists, populate geq[x, 0],
% in the hope that later 
% it will be "in the same e-class" as true[].
geq[x, 0] :- abs[x]

geq[numx, 0] := true[] if num(x, numx), x > 0.
geq[xx, 0] := true[] if mul(x, x, xx).
% if x > 0 and y > 0 are both equivalent to true,
% then x + y > 0 is also equivalent to true.
geq[xy, 0] := true[] 
          if add(x, y, xy), 
             geq(x, 0, true[]), 
             geq(y, 0, true[])
geq[xxyy, 0] := true[] if add(mul[x, x], mul[y, y], xxyy).

% ... other reasoning rules...

% if x>0 is equivalent to true,
% every abs[x] in the database should be equivalent to x 
x := abs[x] if geq(x, 0, true[])
```

The above program can be seen as implementing
a small theorem prover in Gogi.
Whenever it sees abs[x], 
a query about `x >= 0` will be issued to the database.
If later `x >= 0` is proven to be equivalent to true,
a distinguished sort value,
`abs[x]` will be put in the same e-class as `x`.

All these rewrite will be very hard to express in egg.

## Ext 3: Functional Dependency Repair

FDs can be violated: 
what if the user introduced two values for the same set of determinant columns? 
In this case, we need to repair the FDs.
We have seen such examples many times in previous sections.
For example, rules like `R[x1, ..., xk] := ...` will add new values to `R` indexed by `x1, ..., xk`,
and it is likely that there are already other tuples with the same prefix `x1, ..., xk`.
These rules may potentially cause violation of functional dependencies.
In general, there are two kinds of violations:

**Case 1.** 
If the dependent column is a sort value, 
Gogi will unify the two sort values later in the iteration.
We can think of a term of a sort in Gogi as a constant in some theories, 
which refers to some element in the model. 
But we don’t know which element it refers to. 
However, by repairing functional dependencies, 
we can get some clues about what the structure will look like.
Consider the following program
```prolog
rel add(expr, expr) -> expr.
rel num(i64) -> expr.

% add the fact 2 + 1, where the last column is auto-generated.
add[num[2], num[1]]. 

% add the fact 2+1, but the last column is add[num[1], num[2]]
% (add[num[1], num[2]] is created on the fly because
% it occurs at the left hand side.)
add(num[2], num[1], 
   add[num[1], num[2]]). 
```

Because now (without repairing) `add[num[1], num[2]]` will contain two rows. 
The functional dependency is violated.
If we think of rewriting under functional dependency as a process of finding a model for the sort,
then what do we learn from this violation? 
We learned that, to respect the functional dependency,
the two sort values must be the same thing!
Therefore 
the expr originally referred by `add[num[2], num[1]]` and 
by `add[num[1], num[2]]` will be treated as the same expr
and no longer be distinguishable in Gogi!
As we will show later, when a Gogi program reaches the fixpoint, 
it produces a valid, minimal model for the relations and the sorts 
such that the rewrite rules and the functional dependencies are both respected.

TODO: mention no global union-find

**Case 2.**
What if the dependent column is a regular type as a Rust struct or an integer?
Well, we also need to unify them, but in a different way.
The idea here is to describe these values with a algebraic structure, 
which in this case is a lattice. 
A lattice has a bottom (means does not exist) and a top (means conflicts).
Similar to [Flix](https://flix.dev/),
lattice values will grow by taking the least upper bound of 
all the violating tuples.
In that sense, Gogi also generalizes Flix 
(as is described in the [PLDI '16](https://dl.acm.org/doi/10.1145/2980983.2908096) paper).

## Ext 4: Seamless Interop with Rust

This proposed extension takes inspiration 
from recent work on [Ascent](https://dl.acm.org/doi/pdf/10.1145/3497776.3517779),
an expressive Datalog engine that has seamless integration
with the Rust ecosystem.
One interesting feature of Ascent is that it allows 
first-class introspection of the column values.
Ascent use this feature to support features like first-class environment
(this and the next example are both from page 4 of the Ascent paper; comments are mine):
```rust
sigma(v, rho2, a, tick(e, t, k)) <--
 sigma(?e@Ref(x), rho, a, t), // the environment rho is enumerated here
 store(rho[x], ?Value(v, rho2)), // rho[x] is used as an index for store
 store(a, ?Kont(k));
```

One thing though is that Ascent allows enumerating structs as a relation
with the `for` keyword. For example:
```rust
sigma(v, rho2, store, a, tick(v, t,k)) <--
 sigma(?Ref(x), rho, store, a, t),
 // enumerating store[&rho[x]]
 for xv in store[&rho[x]].iter(), if let Value(v,rho2) = xv,
 // enumerating store[a]
 for av in store[a].iter(), if let Kont(k) = av;
```
This makes Ascent have a more macro-y vibe,
which makes sense since the whole Ascent frontend is based
on Rust's procedural macros.
However, I think the similar can be easily achieved 
inside the relational land,
so in a full-fledged relational language like Gogi,
the `for` syntax may not be necessary.

Seamless interop with Rust is in general very powerful.
In fact, we have already used this feature a lot.
For example,
lattices in Gogi are structs defined in Rust 
that implements certain traits.
So rules like `hi(x, n.into()) :- num(n, x).`, 
will call methods 
in the corresponding struct (e.g., `n.into()`).

In general, these user-defined functions 
introduced functional dependencies from domains of functions to their range.
For example,
rule `hi(x, n.into()) :- num(n, x).` can be viewed as
`hi(x, n_into) :- num(n, x), into_rel(n, n_into)` 
with functional dependency from `n` to `n_into`.
Advanced join algorithms
like worst-case optimal joins
can leverage these functional dependencies
to optimize the query.

# The Model Semantics of Gogi and its Evaluation

In this section, we will focus on the problem of how to formalize Gogi 
and how to evaluate Gogi programs. 
This section will first give the model semantics of Gogi.
Then, It will describe
rebuilding, an essential procedure for 
evaluating and maintaining e-graphs,
namely rebuilding, in the Gogi setting.
Finally,
we will discuss how Gogi's matching procedure
can benefit from semi-naive evaluation,
a classic evaluation algorithm in Datalog
## The Model Semantics

For simplicity, we assume that 
in our core language
there will only be one (interpreted) lattice $L$
and one (uninterpreted) sort $S$.

A relation declaration in core Gogi has the following shape:

$$\text{rel }R(c_1,\dots,c_p)\rightarrow (s_1,\ldots, s_m, l_1,\ldots, l_n)$$
where $s_j$ is a sort value column, 
$l_k$ is a lattice value column, 
and $c_i$ can be either a lattice or sort value column. 
Such declaration specifies 
a relation with schema  
$(c_1, \ldots c_p, s_1, \ldots, s_m, l_1, \ldots, l_n)$
and implies the following logical constraint:

$$
 \forall 
 \vec c,\vec s,\vec l,
 \vec{s'},\vec{l'}.
 R(\vec c, \vec s, \vec l)\land
 R(\vec{c'}, \vec{s'}, \vec{l'})\rightarrow
 \vec s=\vec{s'}\land \vec l = \vec{l'}$$
where $\vec x$ denotes a vector of variables $x_1,\ldots,x_k$.

Note in the core formalization, we don't assign default values to lattices 
and we assume the bottom is the default value, 
and each default value is specified as a rewrite rule.

A rewrite rule in the core looks like follows:
$$
 \exists \vec{z}.R_1(\vec{x_1}), \ldots, R_n(\vec {x_n})
 \gets
 S_1(\vec{y_1}), \ldots, S_m(\vec{y_m}).
$$
All variables $x_{ij}$ in the head are either bound in the body
or existentially quantified.
Importantly, existentially quantified variables in the core
quantifies over sort values and 
must be "inferrable" from the functional dependency,
meaning that they must be a dependent variable 
within some relation atoms.
For example, the following Gogi program is not valid,
because rule `R(1, c)` can be triggered arbitrarily many times,
each with different `c`:
```prolog
sort S.
rel R(i64, S).
R(1, c). % translated from R[1].
```

This "inferrable" constraints can be formalized as
$$\forall i.\exists j. z_i\in \text{dep}(R_j(\vec{x_j})),$$
where $\text{dep}\left(R_j\left(\vec{x_j}\right)\right)$ is the set of dependent variables in atom $R_j(\vec{x_j})$.

The rewrite rule in the core is further translated to the following logical constraint:

$$
 \forall \vec{y}.
 \left(\bigwedge_i \vec{y_i}\sqsubseteq_L S_i \rightarrow
 \exists \vec{z}\in S^k. \bigwedge_i \vec{x_i}\sqsubseteq_L R_i\right),$$
<!-- ```prolog
forall y11, ..., yn’mn’:
 (/\i Si(yi1, ..., yim'i)) ->
 exists z1 in L, ..., zk in L,
   /\i (xi1, ... ximi) subset_L Ri
``` -->
where $\vec{y}$ is the set of variables occurring in $\vec{y_1},\ldots, \vec{y_m}$ and
$$
(\vec{c},\vec{s},\vec{l})
\sqsubseteq_L R\iff
\exists \vec{l'}.
R(\vec{c},\vec{s},\vec{l'}) \land\left(\bigwedge_i\;l_i\leq_L l'_i\right)
$$
<!-- ```
(s1, ..., sm, l1, ..., ln) subset_L R
   iff
exists l'1, ..., l'n: 
 R(s1, ..., sm, l'1, ..., l'n) /\
 /\j (lj <= l'j)
``` -->

In English, 
whenever a valuation of variables
satisfied right-hand side, 
there exists some $\vec{z}$ 
such that the left-hand side also "holds",
in the sense that some tuples in the relation
subsumes the substituted left-hand side.

The result of evaluating a (core) Gogi program is 
the minimal model $(S_{\min}, D_{\min})$ that 
satisfies the logical constraints derived from the program,
where $S_{\min}$ is the minimal $S$ of sort values (up to isomorphism)
and $D_{\min}$ is the minimal database instance (i.e., interpretation of relations) with domain $L$ and $S_{\min}$.

This core formalization should look familiar 
to people who know the [chase](https://dl.acm.org/doi/10.1145/3034786.3034796): 
Functional dependencies are equality-generating dependencies (EGD),
and rewrite rules are tuple-generating dependencies (TGD).
However, there are several critical distinctions between Gogi and the chase.
First, the chase has both labelled nulls and constants, 
and unifying two constants will cause a conflict.
Sort values in Gogi can be thought as labelled nulls, 
and there is no matching concept for constants in Gogi.
Moreover,
Gogi supports lattice values.
This feature has its root in both egg's e-class analyses and relational languages like Flix,
and is necessary for different kinds of analysis tasks
that Gogi strives to support.
Finally, and perhaps most importantly,
Gogi has this "inferrable" constraint for existential variables.
This constraint leads Gogi 
to have a single very efficient application algorithm
of executing both EGDs (from functional dependencies) 
and TGDs (from rewrite rules).
In contrast, 
EGDs and TGDs in the chase are executed independently.
Without the lattice values and lifting the inferrable constraint,
Gogi programs can be expressed in the chase.

## The Evaluation Algorithm

The evaluation algorithm of Gogi programs 
consists of two parts.
The core of the evaluation is the invariant-maintaining rebuilding algorithm,
which is inspired 
both by the rebuilding algorithm of egg and 
by the evaluation algorithm of the chase.
The second part involves matching and applying Gogi rules.
Applying Gogi rules is efficient.
In the chase's terminology, 
thanks to the above mentioned inferrable constraint, 
rule application in Gogi is able to utilize functional dependencies
to avoid to generate unnecessary nulls.
Moreover,
because Gogi programs
are monotonic computations over the relational database in nature,
they can benefit from the semi-naive evaluation algorithm of Datalog.
We call this semi-naive matching, which can be seen as 
a further improvement over
[relational e-matching](https://dl.acm.org/doi/10.1145/3498696).


### Rebuilding

The rebuilding algorithm:

```python
todo = mk_union_find()
domain = mk_set()

def union_sort(s1, s2):
 todo.union(s1, s2)
 domain.add_all([s1, s2])

def refresh_todo():
 todo = mk_union_find()
 domain = mk_set()

def on_insert(R, tup):
 # find the tuple by its determinant columns
 orig_tup = R.find_by_determinant(tup.det)
 if orig_tup is None:
   R.insert(tup)
 else:
   # enumerate each dependent column
   for c1 in tup.dep:
     col = c1.col
     c2 = orig_tup[col]
     if col.is_sort():
       s1 = todo.get_or_create(c1)
       s2 = todo.get_or_create(c2)
       union_sort(s1, s2)
     else:
       orig_tup.set_col(col, c1.lat_max(c2))

def normalize(tuple, union_find):
 return tuple.map(lambda val: 
   union_find.get_or_default(val, default = val))

def rebuild(DB):
 while not todo.is_empty():
   # take todo into the local scope
   union_find = todo
   refresh_todo()

   to_remove = mk_set()
   to_insert = mk_set()

   for val in domain:
     for R in DB:
       for col in R.cols:
         for tup in R.index_by(col = col, val = val):
           new_tup = normalize(tup, union_find)
           if new_tup != tup:
             to_remove.add((R, tup))
             to_insert.add((R, new_tup))

   DB.remove_all(to_remove)
   # may trigger on_insert
   DB.insert_all(to_insert)
```

### Applying rewrite rules

```python
def batch_rewrite(pats, DB):
 to_insert = mk_set()
 for (lhs, rhs) in pats:
   for subst in match(DB, lhs):
     subst = chase(DB, subst, lhs, rhs)
     for (R, atom) in rhs:
       to_insert.add((R, atom.apply(subst)))
 DB.insert_all(to_insert)
 return to_insert.is_empty()

def chase(DB, subst, lhs, rhs):
 shouldContinue = True
 while shouldContinue:
   shouldContinue = False

   for atom in rhs:
     det_vars = atom.get_det_vars()
     if det_vars.is_subset_of(subst.get_domain()):
       shouldContinue = True
       
       R = DB.get_rel(atom.rel)
       det = det_vars.apply(subst)
       tup = R.find_by_determinant(det)
       for var in atom.get_dep_vars():
         col = var.col
         if var.is_sort():
             if tup is None: continue
             value = tup.get_by_col(col)
             sort_update(subst, var, value)
         else:
           value = tup is None ? col.lat_init(det)
                               : tup.get_by_col(col)
           lat_update(subst, var, value)

 for var in rhs.get_all_vars():
   if !subst.contains(var):
     assert var.is_sort():
     subst[var] = new_sort_value(var.sort)

def lat_update(subst, var, value):
 if subst.contains(var):
   subst[var] = subst[var].lat_max(value)
 else:
   subst[var] = value

def sort_update(subst, var, value):
 if subst.contains(var):
   union_sort(subst[var], value)
 else:
   subst[var] = value
```
<!-- 
Note to myself: 
This part differs slightly from egglite.
E.g., in egglite, a temp relation is built with all vars.
Here we only accumulate a to_insert list.
We can even stream process everything, so tuples to be
inserted are inserted immediately.

There's also a design choice about how to handle the case when
multiple values exist for the same atom. 
egglite will pick random one, which is correct because 
at the end of the day they will be unified.
However, is there a smarter way of doing this?
Potentially in combination with how we do unify-- because
there still seems to be some populate-then-unify redundancy.

[BELOW ARE OBSOLETE]
the application algorithm here is different from the one
in egglite. In egglite, a temp relation is built.
I think this is because the operator sqlite supports / the sql
language is not rich enough so we can't stream process everything.
Here we can.

They also differ in how they handle the case when multiple
values exist for the same atom (egglite will pick random one
and wait the rebuilding to resolve the violation, Gogi will
??? TODO), 
-->

### The main algorithm

```python
def run(pats, DB, max_iter):
 for iter in range(max_iter):
   if !batch_rewrite(pats, DB):
     return
   rebuild(DB)
```


### Semi-Naive Matching

One of the bottleneck in evaluating Gogi programs
is matching the left-hand side.
Since we are matching over a relational representation
of the e-graphs,
we are already doing is already relational e-matching.
However, we can go one step further:
Let `DB'` be the database of tuples 
that are not touched in the current iteration of rewrite.
`DB'` by itself will not produce any interesting new tuples;
it has to join with newly generated tuples (i.e., the delta database).
This is exactly the semi-naive evaluation algorithm of Datalog.
We call this similar optimization in Gogi semi-naive matching.
This optimization will be tricky to do over e-graph's DAG representation,
yet is fairly obvious in Gogi's full-fledged relational representation.

# Gogi by Example

## Lambda Calculus

```prolog
sort term.
rel false() -> term.
rel true() -> term.
rel num(i32) -> term.
rel var(string) -> term.
rel add(term, term) -> term.
rel eq(term, term) -> term.
rel lam(string, term) -> term.
rel let(string, term, term) -> term.
rel fix(term, term) -> term.
rel cond(term, term, term) -> term.

rel free(term, string).
rel const_num(term, i32).
rel const_bool(term, bool).
rel is_const(term).

% constant folding
const_num(c, i) :- num(i, c).
const_num[c] := const_num[a] + const_num[b] 
            if c = add[a, b].
const_bool[true[], true].
const_bool[false[], false].
const_bool[c] := true[]
             if c = eq[a, a].
const_bool[c] := false[]
             if c = eq[a, b],
                const_num[a] != const_num[b].
is_const(c) :- const_num[c].
is_const(c) :- const_bool[c].

% free variable analysis
free(c, v):- var(v, c).
free[c] := free[a] if c = add[a, b].
% unfolds to free(c, v) :- free(a, v), if c = add[a, b].
free[c] := free[b] if c = add[a, b].
free[c] := free[a] if c = eq[a, b].
free[c] := free[b] if c = eq[a, b].
free(c, v) :- free(body, v) 
          if c = lam[x, body],
             v != x.
% fv(let(x, a, b)) = free(a) + (free(b) \ x)
free(c, v) := free(b, v) 
          if c = let[x, a, b],
             v != x.
free[c] := free[a]
          if c = let[x, a, b].
free(c, v) :- free(body, v) 
          if c = fix[x, body],
             v != x.
free[c] := free[pred] if c = cond[pred, a, b]
free[c] := free[a] if c = cond[pred, a, b]
free[c] := free[b] if c = cond[pred, a, b]

% if-true
then := cond[true[], then, else].
% if-false
else := cond[false[], then, else].
% if-elim
else := cond[eq[var[x], e], then, else]
    if let[x, e, then] = let[x, e, else]
let[x, e, then] :- cond[eq[var[x], e], then, else]
let[x, e, else] :- cond[eq[var[x], e], then, else]
% add-comm
add[b, a] := add[a, b]
% add-assoc
add[a, add[b, c]] := add[add[a, b], c]
% eq-comm
eq[b, a] := eq[a, b]
% fix
let[v, fix[v, e], e] := fix[v, e]
% beta
let[v, e, body] := app[lam[v, body], e]
% let-app
app[let[v, e, a], let[v, e, b]] := let[v, e, app[a, b]]
% let-add
add[let[v, e, a], let[v, e, b]] := let[v, e, add[a, b]]
% let-eq
eq[let[v, e, a], let[v, e, b]] := let[v, e, eq[a, b]]
% let-const
c:= let[v, e, c] if is_const(c)
% let-if
cond[let[v, e, pred], let[v, e, then], let[v, e, else]] 
   := let[v, e, cond[pred, then, else]]
% let-var-same
e := let[v1, e, var[v1]]
% let-var-diff
var[v2] := let[v1, e, var[v2]] if v1 != v2
% let-lam-same
lam[v1, body] := let[v1, e, lam[v1, body]]
% let-lam-diff
lam[v2, let[v1, e, body]] := let[v1, e, lam[v2, body]] 
                         if v1 != v2, free(e, v2)
% capture-avoiding subst
lam[fresh, 
 let[v1, e, 
   let[v2, var[fresh], 
       body]]] 
 := 
let[v1, e, lam[v2, body]] 
 if v1 != v2, !free(e, v2), fresh = gensym().
```

## Type Inference for HM Type System

```prolog
sort expr.
(decl-ctor lam (string expr) expr)
(decl-ctor var (string) expr)
(decl-ctor app (expr expr) expr)
(decl-ctor let (string expr) expr)
(decl-ctor unit () expr)

(decl-sort type)
(decl-ctor tvar (string) type)
(decl-ctor tunit () type)
(decl-ctor tarr (type type) type)
; we don't need to introduce additional
; rules for case like ((tvar s), tunit),
; where we will normally subst one to the other,
; because here once union! is performed,
; the object language can't distinguish
; (tvar s) from tunit
(rule (
   (eq (tarr fr1 to1) (tarr fr2 to2)
)(
   (union! fr1 fr2)
   (union! to1 to2)
))


; a ctx is a list of var -> scheme
(decl-sort ctx)
(decl-ctor nil () ctx)
(decl-ctor cons (string scheme) ctx)

; a scheme is a type with some type variables qualified
(decl-sort scheme)
(decl-ctor forall (string-list, type) scheme)


; \Sigma |- e : t
(decl-fun type-of 
   (: (ctx expr) type)
   (default \bot) ; in default case the type is None
                  ; each bottom value should be unique
                  ; so they don't accidentally merge
   (merge union!)) ; when merge, unify the type

(decl-fun generalize (: (type) scheme) TODO)
(decl-fun instantiate (: (scheme) type) TODO)

; \Sigma |- () : Unit
(merge (type-of ctx unit) tunit)

; \Sigma; x: t1 |- e : t2
; -------------------------------
; \Sigma |- lam x. e : t1 -> t2
(rule (
   (eq a (type-of ctx (lam x e)))
   (eq s (fresh-str))
)(
   (merge a (tarr (tvar s) 
                  (type-of (cons x (forall [] (tvar s)))
                              e)))
))

; \Sigma |- e1 :- t1 -> t2
; \Sigma |- e2 : t1
---------------------------------
; \Sigma |- e1 e2 : t2
(rule (
   (eq a (type-of ctx (app e1 e2)))
   (eq s (fresh-str))
)(
   (union! (type-of ctx e1) 
           (tarr (type-of ctx e2)
                 (tvar s)))
   (union! a (tvar s))
))


(rule (
   (eq a (type-of ctx (let x e1 e2)))
)(
   (union! a
           (type-of (cons x (generalize (type-of ctx e1)) 
                          ctx) 
                       e2))
))
```

# Comparison to other languages

## Comparison to Rel

## Comparison to Souffle

## Comparison to Flix

## Comparison to the Chase
% \input{chapters/05-formal.tex}
% \input{chapters/06-eval.tex}
\chapter{Related works}

\section{E-graphs and data structures for program representation}

\Egraphs are first introduced in Greg Nelson's seminal thesis \citep{nelson-thesis}
 in the 1970s 
 as a way of effectively deciding the theory of equalities.
A more efficient algorithm is introduced by \citet{tarjan-congruence} 
 and the time complexity of this algorithm is analyzed.
\Egraphs are then used at the core of 
 various theorem provers and solvers \citep{simplify, z3, cvc4}.
In the 2000s, 
 \egraphs are repurposed for program optimization \citet{eqsat,denali}.
The technique, known as equality saturation, 
 repeatedly performs non-destructive rewriting on the \egraphs 
 to grow a compact space of equivalent programs.
An extraction procedure is then applied to extract the optimal program.
In essense, equality saturation mitigates the phase-ordering problem by keeping 
 all programs.
This inspires later work on using \egraphs for
 translation validation \citep{eqsat-tv}, 
 floating-point arithematic \citep{herbie},
 semantic code search \citep{semsearch},
 and computer-aided design \citep{carpentry-compiler}.
However, a general framework for \egraphs is not yet available,
 and developing applications using \egraphs is a tedious effort.
It is not until early 2020s that a generic framework for \egraphs, 
 called \texttt{egg}, is developed \citep{egg}.
As a result, many projects sprang up building domain-specific projects using \egraphs,
 including rewrite rule synthesis \citep{ruler}, machine learning compiler \citep{tensat,glenside},
 relational query optimization \citep{spores}, and
 digital signal processing vectorization \citep{diospyros}.



Many works on \egraphs focus on improving the efficiency and usability of \egraphs.
Some of these works are subsumed by the relational \egraphs.
For example, \citep{efficient-ematching} studies 
 the incremental maintenance problem of the \ematching procedure
 and one of its standard extension called multi-patterns.
\citet{tensat} also proposed an algorithm 
 to extend \egraph frameworks with multi-patterns for equality saturations.
In relational \egraphs, multi-patterns are supported naturally \citep{relational-ematching},
 and incremental maintenance can be achieved using semi-na\"ive evaluation,
 a standard database technique \citep{datalog-survey}.
For some other works, 
 the problem of finding and understanding its relational dual 
 is still open to future research.
For instance,
 applications like SMT solvers not only want to know if two terms are equivalent, 
 but also why they are equivalent.
Techniques are developed to generate proofs for equivalences in \egraphs \citep{proof-producing}.
It is speculated that proofs for congruence closure in relational form 
 may just be database provenance \citep{prov-semiring,prov-souffle}.
Moreover,
 scheduling is a critical component of equality saturation \citep{egg}, 
 and a good scheduling algorithm is a key enabler of scalable equality saturations.
However, 
 for relational languages like Datalog,
 the scheduling problem is less studied,
 because Datalog programs are usually run until fixpoint,
 while the fixpoint for equality saturation is usually infinitary.
It is therefore a future direction to study the scheduling problem for relational \egraphs.
% The wide variety of e-graph applications is also placing new
%  demands on the capability of e-graph frameworks. 
% For example,
%  many \egraph--based applications 
%  use a standard extension to e-matching called multi-patterns \citep{tensat,efficient-ematching}. 
% Efficient support for
%  multi-patterns requires complicated modification of the basic backtracking algorithm provided. 
% % Most existing frameworks either do
% %  not support multi-patterns or support them inefficiently. 
% Practical applications may also interleave equational reasoning
%  with non-equational ones. 
% However, non-equational reasoning like
%  logical implication is fairly non-trivial and potentially inefficient in
%  existing e-graph frameworks like egg.

The connection between \egraphs and relational databases.
 is first studied in our earlier work on relational \ematching\citep{relational-ematching}.
In relational \ematching, 
 we proposed to view an \egraph as a relational database,
 which allows us to make \ematching orders of magnititude faster
 and prove desired theoretical properties.
However, to use this technique,
 one has to keep both the \egraph and its relational representation 
 and convert back and forth, which limits its practical adoptions.
We build on this work, 
 which only takes a static relational snapshot of \egraph each time,
 and explore how \egraphs as a relational database will behave dynamically.
This saves us from the labor of keeping and syncing between the two \egraph representations
 and further exploits the benefits of the relational \ematching approach.

Other data structures for compact program representation 
 have also been long studied in the literature for decades,
 version space algebras (VSAs) \citep{vsa,flashmeta} 
 and finite tree automata \citep{blaze, dace} in particular.
Recently, it is shown that both \egraphs and VSAs are special cases of finite tree automata \citep{vsa-eq-fta}.
A natural question is therefore if we can encode VSAs and finite tree automata relationally
 and how we can possibly benefit from such an encoding for tasks 
 like program synthesis and program optimization.

\section{Relational databases and Datalog}

Our work is closely related to works on Datalog and relational database.
Relational \egraphs are directly inspired by work on the chase 
 and work on recursive aggregates for Datalog.
Data dependencies describe dependencies between columns in a relational database, 
 and the chase is a family of iterative algorithms
 for reasoning about data dependencies \citep{chase-revisited, bench-chase}.
We showed that congruence and rewrites in equality saturation 
 can be described using data dependencies, 
 and therefore equality saturation can be effectively viewed as a chase algorithm.
As a chase, equality saturation has nice properties that are worth further study:
While there may be many finite universal models to data dependencies in the chase,
 in equality saturation,
 there will be at most one finite universal model, 
 which is the core.
Moreover,
 equality saturation terminates 
 if and only if there is a finite universal model of the given dependencies, 
 which equality saturation will output,
In contrast,
 the (non-core) chase does not necessarily terminate 
 even when a finite universal solution of the input dependencies exists,
 or such a finite solution is very expensive to compute (with the core chase \citep{chase-revisited}).
As a future direction,
 we would like to further understand equality saturation using the theories developed 
 in database research.

% Flix, subsumption

% Datafun

% Rel, provenance semiring, datalogo

% bloom: Logic and lattices for distributed programming

% \section{Chase}

% Database exchange, certain answer, query optimization, ...

% \section{Satisfiability Modulo Theories}

% Our design of the interface is inspired by SMT.

% \section{Logic programming}

% Constraint Handling Logic (where you can encode union find)


\bibliographystyle{ACM-Reference-Format}
\bibliography{main}

\end{document}
